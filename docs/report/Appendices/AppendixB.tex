% Appendix B

\chapter{Enigma Protocol Specification}
\label{AppendixB}
\lhead{Appendix B. \emph{Enigma Protocol Specification}}

\section{Introduction}

\section{Terminology}

\section{Definitions}

\section{Format}

\section{Commands}

\subsection{Connection}

A connection is represented by a streaming XML document, with the root element being \verb!<stream>!.

\begin{itemize}
	\item \textbf{Opening a connection:}
		\begin{verbatim}
			<stream  to="SERVER_NAME"
				         from="USER_NAME"
				         id="SESSION_ID"
				         return-port="LOCALHOST_INBOUND_PORT"
				         xmlns="enigma:client">
		\end{verbatim}
	\item \textbf{Closing a connection:}
		\begin{verbatim}
			</stream>
		\end{verbatim}
\end{itemize}

\subsection{Authentication}

\begin{itemize}
	\item \textbf{Toggling encryption:}
		\begin{verbatim}
			<auth stage="streaming"
			      id="SESSION_ID"
			      type="toggle">
			      [off|on]
			</auth>
		\end{verbatim} \\
		Note: this requires the agreement of both users. The connection should be closed if one user disagrees to change the current encryption status.
\end{itemize}

\begin{itemize}
	\item \textbf{Asserting the key agreement method:}
		\begin{verbatim}
			<auth stage="agreement"
			      id="SESSION_ID"
			      type="method">
			      [method type identifier]
			</auth>
		\end{verbatim} \\
\end{itemize}

\begin{itemize}
	\item \textbf{Publishing a certificate:}
		\begin{verbatim}
			<auth stage="agreement"
			      id="SESSION_ID"
			      type="cert">
			      [Base64 encoded Enigma Certificate]
			</auth>
		\end{verbatim} \\
\end{itemize}

\begin{itemize}
	\item \textbf{Publish an encrypted symmetric cipher key:}
		\begin{verbatim}
			<auth stage="agreement"
			      id="SESSION_ID"
			      type="key"
			      [method="CIPHER_ALGORITHM"]>
			      [Base64 encoded encrypted key]
			</auth>
		\end{verbatim} \\
\end{itemize}

\subsection{Messaging}

\begin{itemize}
	\item \textbf{Sending a message:}
		\begin{verbatim}
			<message  [to="REMOTE_USER_NAME"]
				          from="USER_NAME"
				          id="SESSION_ID"
				          [type=""]>
			    <body>MESSAGE_CONTENT</body>
			</message>
		\end{verbatim}
\end{itemize}

\subsection{Errors}

Errors do not have a specific element themselves, but are included as a subelement of any other tag, setting \verb!att:type! to \verb!error!.

\begin{itemize}
	\item \textbf{Sending an error:}
		\begin{verbatim}
			<element  [other attributes]
				          type="error">
			    <error type="ERROR_NUMBER">
			        ERROR_MESSAGE
			    </error>
			    [element contents]
			</element>
		\end{verbatim}
\end{itemize}