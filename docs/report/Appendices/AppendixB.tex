% Appendix B

\chapter{Enigma Protocol Specification}
\label{AppendixB}
\lhead{Appendix B. \emph{Enigma Protocol Specification}}

\section{Introduction}

The Enigma Protocol is a communications protocol intended for use in Instant Messaging applications. It is very loosely based around the concept of \emph{Jabber/XMPP}\footnote{http://xmpp.org/about-xmpp/history/}, primarily as it is an application of the Extensible Markup Language (XML) to allow for near-real-time communications between networked devices. The XMPP protocol is considerably more complex than Enigma, and the two should not be considered comparable.

This document covers the basic XML elements that must be implemented by an Enigma Protocol-based application. As its primary purpose is to provide a simple way of sending text with meta-data, it should only be used for research purposes. The Enigma Application can be considered a reference implementation, however it is not a complete implementation.

\section{Requirements}

An IM application for use in the testing of sending encrypted messages should have the capability to:

	\begin{enumerate}
		\item Be able to exchange brief text messages in near-real-time.
		\item Transmit information that can be used to securely generate and exchange encryption keys.
		\item Transmit information that can be used to identify and authenticate.
	\end{enumerate}
	
And thus, any application implementing the Enigma Protocol must be able to adhere to the above requirements.

\section{History}

This document covers version $1.0$ of the Enigma protocol. There are no prior implementations.

\section{Terminology}

No specialised terminology is used without explanation in this document.

\section{Format}

A conversation between two users should be considered as the building of a valid XML document using the \verb!enigma:client! namespace. The document should consist of a valid root element occurring only once, valid child elements matching those listed in the document, and finish with a closing tag matching the root element. Each packet, that is not a root element, must consist of a start-tag and end-tag, or an empty-element tag otherwise it \textbf{should not} be parsed and \textbf{should} be dropped.

The order of the messages is important if the \verb;time; attribute is not included, in which case they should be handled and displayed in a first-in-first-out order.

\section{Commands}

\subsection{Connection}

A connection is represented by a streaming XML document, with the root element being \verb!<stream>!.

\begin{itemize}
	\item \textbf{Opening a connection:}
		\begin{verbatim}
			<stream  to="SERVER_NAME"
				         from="USER_NAME"
				         id="SESSION_ID"
				         return-port="LOCALHOST_INBOUND_PORT"
				         xmlns="enigma:client">
		\end{verbatim}
	\item \textbf{Closing a connection:}
		\begin{verbatim}
			</stream>
		\end{verbatim}
\end{itemize}

\subsection{Authentication}

\begin{itemize}
	\item \textbf{Toggling encryption:}
		\begin{verbatim}
			<auth stage="streaming"
			      id="SESSION_ID"
			      type="toggle">
			      [off|on]
			</auth>
		\end{verbatim} \\
		Note: this requires the agreement of both users. The connection should be closed if one user disagrees to change the current encryption status.
\end{itemize}

\begin{itemize}
	\item \textbf{Asserting the key agreement method:}
		\begin{verbatim}
			<auth stage="agreement"
			      id="SESSION_ID"
			      type="method">
			      [method type identifier]
			</auth>
		\end{verbatim} \\
\end{itemize}

\begin{itemize}
	\item \textbf{Publishing a certificate:}
		\begin{verbatim}
			<auth stage="agreement"
			      id="SESSION_ID"
			      type="cert">
			      [Base64 encoded Enigma Certificate]
			</auth>
		\end{verbatim} \\
\end{itemize}

\begin{itemize}
	\item \textbf{Publish an encrypted symmetric cipher key:}
		\begin{verbatim}
			<auth stage="agreement"
			      id="SESSION_ID"
			      type="key"
			      [method="CIPHER_ALGORITHM"]>
			      [Base64 encoded encrypted key]
			</auth>
		\end{verbatim} \\
\end{itemize}

\subsection{Messaging}

\begin{itemize}
	\item \textbf{Sending a message:}
		\begin{verbatim}
			<message  [to="REMOTE_USER_NAME"]
				          from="USER_NAME"
				          id="SESSION_ID"
				          [type=""]>
			    <body>MESSAGE_CONTENT</body>
			</message>
		\end{verbatim}
\end{itemize}

\subsection{Errors}

Errors do not have a specific element themselves, but are included as a subelement of any other tag, setting \verb!att:type! to \verb!error!.

\begin{itemize}
	\item \textbf{Sending an error:}
		\begin{verbatim}
			<element  [other attributes]
				          type="error">
			    <error type="ERROR_NUMBER">
			        ERROR_MESSAGE
			    </error>
			    [element contents]
			</element>
		\end{verbatim}
\end{itemize}