\chapter{Introduction}
\label{Chapter1}
\lhead{Chapter 1. \emph{Introduction}}

\section{What it is about}

Secrecy has always been of great importance, not just in modern society, but throughout history. Until very recently, Cryptography was consigned to the sending and receiving of messages, generally using pen and paper. However, increasingly cryptography -- the study and implementation of techniques for secure communication -- is becoming more vital to the smooth running of even basic systems, whether it is the clich\'{e} example of military secrets or simply a micro-payment for an online service. Initially, the usage of provably secure and efficient cryptographic algorithms was limited to governments and related contractors, however the advent of cryptographic standards, and more relevantly the creation of public-key cryptography mechanisms, has pushed it in to a wider field of use as researchers gained a better understanding of the area. 

\section{Goals and Intentions}

The primary goal of this project can be found in the title and abstract: to research, discuss and create algorithms within or related to the field of number theory. To complete this task I will be developing my thoughts and discoveries regarding number theory as this report progresses, as well as producing a number of deliverables in the form of software applications, test results and statistics.The topic of number theory is in itself fascinating, and extraordinarily vast, however I will only be scratching the surface. Nonetheless, I hope to explain throughout this project how number theory has become such an important part of modern cryptography, and what it can be used for.

\section{Project Summary}

I will be splitting the project in to three main parts, excluding this report:

\begin{enumerate}
	\item \textbf{Prime numbers in software} \\
		A discussion of how prime numbers, and an application of number theory, can be efficiently produced programmatically, and the software to prove it.
	\item \textbf{Algorithms} \\
		The development of algorithms using prime numbers, such as RSA, and complementary cryptographic algorithms such as AES. Alongside this, the development of non-major algorithms in the field of prime number based cryptography that still present academically interesting concepts.
	\item \textbf{Final Application} \\
		The production of a software program that utilises the implemented cryptographic algorithms in \textsection\ref{Chapter2} to display their efficacy in a real world application.
\end{enumerate}

The primary algorithms to develop are:

\begin{itemize}
	\item Asymmetric
		\begin{itemize}
			\item RSA
			\item Diffie-Hellman
		\end{itemize}
	\item Symmetric
		\begin{itemize}
			\item AES
		\end{itemize}
	\item Identification and Authentication
		\begin{itemize}
			\item RSA Signing
			\item Digital Certificates
		\end{itemize}
\end{itemize}

The reasoning behind this selection is thus: popularity. The best algorithms are those that are mature and have been scrutinised thoroughly by an open source community. However, we will further discuss this choice as the report progresses. Other algorithms will be discussed and produced alongside these, however they will not be used in the final application testbed and are purely for research interest. These can be found listed in the contents in the relevant sections.
		
\section{Other Information}

This document was written and composed with \LaTeX \ using \emph{Taco Software's} Latexian\footnote{http://tacosw.com/latexian/}. The \LaTeX \ source code for this report should have come bundled with the project documentation and files, however if you are unable to retrieve it please see \emph{\textsection \ref{sec:project_repo}}.

\emph{Eclipse Indigo}\footnote{http://eclipse.org} was used for Java development, the primary language used in this project, along with \emph{MacVim}\footnote{http://code.google.com/p/macvim/} for general source and text editing.

Git with \emph{GitHub}\footnote{http://github.com} was used for source control, with \emph{GitHub Issues} used for bug tracking. \emph{Trello}\footnote{http://trello.com} was used for idea and to-do management. \emph{OmniGraffle}\footnote{http://www.omnigroup.com/products/omnigraffle/} was used to produce graphics, such as sequence diagrams and system overviews.

\subsection{A note on openness}

As with any field of study, the quality of research and development is dependent on the open distribution and sharing of ideas. This is \emph{particularly} important with regards to cryptography. As said, cryptography was once reserved to government and research was conducted in secrecy. The open sharing of relevant information in this field is not just for the furthering of knowledge, but also to allow others to inspect and examine algorithms, a process that drastically improves the security of a system. As such, the entirety of this project is licensed under the \textbf{GNU Lesser General Public License version 3 or greater} and is available for access publicly online.

\subsection{Project Repository}
\label{sec:project_repo}

\textbf{If any part of this report is missing, you believe that you do not have a full access to the source code discussed in this document, or if any files have been lost, it is available for full download at \url{http://cyanoryx.com/files/project.tgz}. It is also available at \url{http://github.com/adammulligan/Enigma}.}

Throughout the report we reference files by their package locations. For example, \emph{com.cyanoryx.uni.crypto.rsa.RSA\_OAEP} signifies that the file is located at \emph{src/com/cyanoryx/uni/crypto/RSA\_OAEP.java}.