% Chapter 7

\chapter{Outcomes} 
\label{Chapter7}
\lhead{Chapter 7. \emph{Outcomes}} 

\section{Fulfilment of Specification}

All points in the specification (see \textsection\ref{AppendixA}) were met fully, and the requirements have been fulfilled entirely. The primary purpose has been fulfilled: to create an application that will allow the testing of the efficacy of cryptographic algorithms within a real-life application.

\section{Comparison to JDK}

  \subsection{RSA}
  
    \begin{center}
      \begin{tabular}{ | l | p{4.0cm} |}
        \hline
        Library & Average Time (20 iterations; nanoseconds) \\ \hline \hline
        Enigma & 6662350 \\ \hline
        JDK & 445450 \\ \hline \hline
        \textbf{Difference} & 6216900 \\
        \hline
      \end{tabular}
    \end{center}
  
  \subsection{Symmetric Cryptography (AES)}
  
    \begin{center}
      \begin{tabular}{ | l | p{4.0cm} |}
        \hline
        Library & Average Time (20 iterations; nanoseconds) \\ \hline \hline
        Enigma & 293350 \\ \hline
        JDK & 60550 \\ \hline \hline
        \textbf{Difference} & 232800 \\
        \hline
      \end{tabular}
    \end{center}
    
  \subsection{Asessment}
  
  As can be seen, the Enigma versions of both algorithms are considerably slower than the JDK versions -- indeed, in both cases an \emph{order of magnitude} slower. Why is this the case? 
  
  \begin{enumerate}
    \item 
  \end{enumerate}

\section{Summary}

Our aim in this project was to produce, from scratch and using only technical documentation, implementations of cryptographic algorithms that could be used in a real world application. This objective was completed successfully, and a large amount of new knowledge has been gained. However, the final recommendation? Do not do this. Cryptographic implementations have long been standardised and produced in mature libraries that are easy to use.

The main point is this: if you are doing something so unusual with cryptography such that you need to implement algorithms yourself, then you shouldn't be doing that. If you aren't doing something unusual, then you can use the libraries already available.

This brings us back to the quote from Bruce Schneier given at the beginning of this report: 

\textit{``Anyone can design a security system that he cannot break. So when someone announces, ``Here’s my security system, and I can`t break it,” your first reaction should be, “Who are you?” If he`s someone who has broken dozens of similar systems, his system is worth looking at. If he`s never broken anything, the chance is zero that it will be any good.''}

Non-standard and custom implementations are only truly safe when they have been created by developers experienced in the production of cryptosystems and tested thoroughly.

The author hopes that this report has been of interest to you, or at least beneficial to your knowledge of information security, cryptosystems and how difficult it really is to get it right.