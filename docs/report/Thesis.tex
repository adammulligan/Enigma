%% ----------------------------------------------------------------
%% Project Report - Adam Mulligan
%%
%% Template based on http://www.sunilpatel.co.uk/thesis-template/
%% ---------------------------------------------------------------- 

\documentclass[a4paper, 11pt, oneside]{Thesis}  % Use the "Thesis" style, based on the ECS Thesis style by Steve Gunn
\graphicspath{{Figures/}}  % Location of the graphics files (set up for graphics to be in PDF format)

\usepackage[square, numbers, comma, sort&compress]{natbib}  % Use the "Natbib" style for the references in the Bibliography
\usepackage{verbatim}  % Needed for the "comment" environment to make LaTeX comments
\usepackage{vector}  % Allows "\bvec{}" and "\buvec{}" for "blackboard" style bold vectors in maths
\usepackage{amsthm}
\hypersetup{urlcolor=black, colorlinks=false}

%% =====================
% SETUP & TITLE PAGE

\begin{document}
\frontmatter

\title  {Enigma: Prime Numbers and Cryptosystems}
\authors  {\texorpdfstring
            {\href{http://cyanoryx.com}{Adam Mulligan}}
            {Adam Mulligan}
            }
\addresses  {\groupname\\\deptname\\\univname} 
\date       {\today}
\subject    {}
\keywords   {}

\maketitle

%% =====================
% STYLES

\setstretch{1.3}  

\fancyhead{}  
\rhead{\thepage}  
\lhead{}  

\pagestyle{fancy}  


%% =====================
% QUOTE

\pagestyle{empty} 

\null\vfill
\textit{``Anyone can design a security system that he cannot break. So when someone announces, “Here’s my security system, and I can’t break it,” your first reaction should be, “Who are you?” If he’s someone who has broken dozens of similar systems, his system is worth looking at. If he’s never broken anything, the chance is zero that it will be any good.''}

\begin{flushright}
Bruce Schneier, \textit{The Ethics of Vulnerability Research}
\end{flushright}

\vfill\vfill\vfill\vfill\vfill\vfill\null
\clearpage 

%% =====================
% ABSTRACT

\addtotoc{Abstract} 

\abstract{
	\addtocontents{toc}{\vspace{1em}}  % Add a gap in the Contents, for aesthetics

	Modern cryptography allows us to perform many types of information exchange over insecure 
	channels. One of these tasks is to agree on a secret key over a channel where messages can 
	be overheard. This is achieved by Diffe-Hellman protocol. Other tasks include public key 
	and digital signature schemes; RSA key exchange can be used for them. These protocols are 
	of great importance for bank networks.

	Most such algorithms are based upon number theory, namely, the intractability of certain 
	problems involving prime numbers. The project involves implementing basic routines for 
	dealing with prime numbers and then building cryptographic applications using them.
}

\clearpage 

%% =====================
% ACKNOWLEDGMENTS

\setstretch{1.3}  % Reset the line-spacing to 1.3 for body text (if it has changed)

% The Acknowledgements page, for thanking everyone
\acknowledgements{
\addtocontents{toc}{\vspace{1em}}  % Add a gap in the Contents, for aesthetics

With thanks to my project advisor Yuri Kalnishkan, and the RHUL Department of Computer Science for three years worth of knowledge and experience.

}
\clearpage  % End of the Acknowledgements

%% =====================

\pagestyle{fancy} 

%% =====================

\lhead{\emph{Contents}}  
\tableofcontents  % Write out the Table of Contents

%% =====================

\lhead{\emph{List of Figures}} 
\listoffigures  % Write out the List of Figures

%% =====================

\lhead{\emph{List of Tables}}  
\listoftables  % Write out the List of Tables

%% =====================
% ABBREVIATIONS

\setstretch{1.5}  
\clearpage  

\lhead{\emph{Abbreviations}}  

\listofsymbols{ll}  {
	\textbf{AES} & \textbf{A}dvanced \textbf{E}ncryption \textbf{S}tandard \\
}

%% =====================
% CHAPTERS

\mainmatter
\pagestyle{fancy}  

\chapter{Introduction}
\label{Chapter1}
\lhead{Chapter 1. \emph{Introduction}}

\section{What it's about}

Secrecy has always been of great importance, not just in modern society, but throughout history. Until very recently, Cryptography was consigned to the sending and receiving of messages, generally using pen and paper. However, increasingly cryptography -- the study and implementation of techniques for secure communication\footnote{http://en.wikipedia.org/wiki/Cryptography} -- is becoming more vital to the smooth running of even basic systems, whether it's the clich\'{e} example of military secrets or simply a micro-payment for an online service. Initially, the usage of provably secure and efficient cryptographic algorithms was limited to governments and related contractors, however the advent of encryption standards, and more relevantly, the creation of public-key cryptography mechanisms, has pushed it in to a wider field of use as researchers gained a better understanding of the area. 

\section{Goals and Intentions}

The primary goal of this project can be found in the title and abstract: to research, discuss and create algorithms within or related to the field of prime numbers. To complete this task I will be developing my thoughts and discoveries regarding prime numbers as this report progresses, as well as producing a number of deliverables in the form of software applications, test results and statistics.The topic of number theory is in itself fascinating, and extraordinarily vast, however I will only be scratching the surface. Nonetheless, I hope to explain throughout this project how prime numbers have become such an important part of modern cryptography, and what they can be used for.

\section{Project Structure}

I will be splitting the project in to four main parts, excluding this report:

\begin{enumerate}
	\item \textbf{Prime numbers in software} \\
		A discussion of how prime numbers can be efficiently produced programmatically, and software to prove it.
	\item \textbf{Algorithms} \\
		The development of algorithms using prime numbers, such as RSA, and complementary cryptographic algorithms such as AES. Alongside this, the development of non-major algorithms in the field of prime number based cryptography that still present an academically interesting concept.
	\item \textbf{Final Application} \\
		The production of a software program that utilises the implemented cryptographic algorithms in section 2 to display their efficacy in a real world application.
	\item \textbf{Cryptanalysis} \\
		Taking the algorithms created and comparing them with official implementations for statistical analysis, alongside researching and performing "attacks" on them to prove or disprove the implementation's cryptographic security.
\end{enumerate}

The primary algorithms to develop are:

\begin{itemize}
	\item Asymmetric
		\begin{itemize}
			\item RSA
			\item Diffie-Hellman
		\end{itemize}
	\item Symmetric
		\begin{itemize}
			\item AES
		\end{itemize}
	\item Identification and Authentication
		\begin{itemize}
			\item RSA Signing
			\item Digital Certificates
		\end{itemize}
\end{itemize}

Other algorithms will be discussed and produced alongside these, however they will not be used in the final application testbed and are purely for research interest. These can be found listed in the contents in the relevant sections.
		
\section{Other Information}

This document was written and composed with \LaTeX using \emph{Taco Software's \href{http://tacosw.com/latexian/}{Latexian}}. The \LaTeX source code for this report should have come bundled with the project documentation and files, however if you are unable to retrieve it please see section~\ref{sec:repo}.

\subsection{Project Repository} \label{sec:repo}

\textbf{If: any part of this report is missing, you believe that you do not have a full access to the source code discussed in this document, or if any files have been lost, it is available for full download at \url{http://cyanoryx.com/files/project.zip} (SHA-1 Sum: x)}.

% Chapter 2

\chapter{Mathematical Basis}
\label{Chapter2}
\lhead{Chapter 2. \emph{Mathematical Basis}}


\section{Number Theory}



\section{Prime Numbers}

\subsection{Finding Prime Numbers}

 

% Chapter 3

\chapter{Number Theory and Public-key Cryptography}
\label{Chapter3}
\lhead{Chapter 3. \emph{Number Theory and Cryptography}} 

\section{Overview}
\section{Integer Factorisation}
\section{Prime Number Generation}
\section{RSA}
\section{Other Variations} 

% Chapter 4

\chapter{Symmetric Cryptography} 
\label{Chapter4}
\lhead{Chapter 4. \emph{Symmetric Cryptography}} 

\section{Introduction}

Symmetric cryptography is a type of \emph{secret-key cryptosystem}, meaning that the encryption and decryption transformations use the same key and the encryption function is one-to-one (and thus invertible). More specifically we can define a symmetric algorithm as a cryptosystem with keys $k_{encrypt}$ and $k_{decrypt}$, where $k_{encrypt} = k_{decrypt}$. This is known as having a shared secret.

Simplified, encryption can be written as:

\begin{center}
  $c = e_k(m)$
\end{center}

Where $c$ is the ciphertext, $e$ the cipher function, $k$ the key, and $m$ the plaintext message. Decryption, this being a \emph{symmetric} cipher, is almost exactly the same:

\begin{center}
  $m = e_k(c)$
\end{center}

Symmetric key algorithms can be implemented in two forms: stream and block ciphers, however we will only be looking at block ciphers. 

\begin{enumerate}
  \item \textbf{Stream Ciphers} encrypt and decrypt data one character at a time.
  \item \textbf{Block Ciphers} differ in that they work on fixed-length groups of data using a transformation.
\end{enumerate}

  \subsection{Differences and Which To Use}
  
  We are considering here fundamental properties of \emph{basic} block and stream ciphers.

    \subsubsection{Block Ciphers}
    
    A block cipher is a function that maps $n$-bit blocks of plaintext to $n$-bit blocks of ciphertext using a transformation function. 
    
    \begin{center}
       \begin{tabular}{ | p{6cm} | p{6cm} |}
          \hline
          Pros & Cons \\ \hline \hline
          It is impossible to insert or modify characters within a block without detection. & Single characters of messages are not able to be encrypted (unless the message comprises solely of one character) without the whole block having been received. \\ \hline
          Frequency analysis impossible. & Error propagation is high (at least, relative to stream ciphers) as one error can affect a whole block. \\
          \hline
        \end{tabular}
      \end{center}
    
    \subsubsection{Stream Ciphers}
    
    A stream cipher maps a single character $n$ to an encrypted character $n'$.
    
    \begin{center}
       \begin{tabular}{ | p{6cm} | p{6cm} |}
          \hline
          Pros & Cons \\ \hline \hline
          Low error propagation. & Low diffusion -- frequency analysis possible. \\ \hline
          & New messages can be constructed by using parts of older messages. \\ 
          \hline
        \end{tabular}
      \end{center}
    
    \subsubsection{Summary}

    Overall, what can we say? The matter is mostly a balance of preference and perceived security. In this case, we will be using a block cipher known as the AES (Advanced Encryption Standard) algorithm, for a number of reasons:
    
    \begin{enumerate}
      \item There are few modern stream ciphers that are well documented compared to AES\footnote{http://www.ecrypt.eu.org/stream/ was an initiative to find modern day stream ciphers for widespread use}
      \item It is the current standard for encryption as defined by the United States National Institute of Standards and Technology, and is recommended by the United States National Security Agency for securing top secret information.
      \item And perhaps most importantly, it is vastly more interesting to implement.
    \end{enumerate}
    
    In reality, block ciphers and stream ciphers offer no security benefit over one another.

\section{AES}
  \subsection{Overview}
  
  AES itself is the name of the \emph{standard} whereas the actual name of the algorithm is known as \emph{Rijndael}, though we will refer to it as the AES algorithm. It is a symmetric block cipher with a block size of 128 bits and a key size of 128, 192, or 256 bits. 
  
  \subsection{Mathematical Preliminaries}
  
  AES mostly utilises basic bitwise arithmetic on fixed-size blocks, and so the mathematics behind it is not particularly complex. An understanding of bitwise operators like XOR, bit shifts, logical operators is required, but all other concepts are explained. However, there is one concept that can be difficult to understand relative to the algorithm without explanation: finite fields.
  
    \subsubsection{Finite Fields}
    
    Finite fields, like most things useful to cryptography, are an important part of number theory alongside cryptography. We have used them previously in the Diffie-Hellman protocol, however they were left with little explanation. 
    
    Finite fields are, perhaps unsurprisingly, a type of field with a finite number of elements and are a subclass of two other algebraic structures:
    
    \begin{enumerate}
      \item A field is a ring of elements that are non-zero and are formed under multiplication.
      \item A ring is a set of elements with only two operations (addition and multiplication), that must follow a set of properties:
      \begin{enumerate}
        \item It is an abelian group under addition.
        \item Under multiplication, it satisfies the closure, associativity and identity axioms.
        \item Elements are commutative.
        \item Elements satisfy the distribution axiom: $a \times (b + c) = a \times b + a \times c$
      \end{enumerate}
    \end{enumerate}
    
    How is this useful to AES? Only two operations are required for our transformations of bytes -- addition (XOR) and multiplication -- and we want to limit our calculations to a finite number of possible elements ($2^8$), both of which a finite field offers. As we will explain later, this also means we can use an irreducible polynomial that limits calculations to be within one byte.
    
  \subsection{Algorithm}
  \label{subsec:aes_algo}
    
    Officially the Rijndael algorithm is a cipher with multiple block and key sizes, however we will only be following the AES standard of a fixed block size (16 bytes) and three key sizes (128,192 and 256 bits).
    
    Given a 16-byte (128 bit) message, which is equal to one block, we have bytes $b$ such that:
    
    \begin{center}
      $\begin{pmatrix}
        b_0 & b_1 & b_2 & b_3 \\
        b_4 & b_5 & b_6 & b_7 \\
        b_8 & b_9 & b_{10} & b_{11} \\
        b_{12} & b_{13} & b_{14} & b_{15}
      \end{pmatrix}$
    \end{center}
    
    The AES algorithm, like most modern symmetric block ciphers, consists of a number of "rounds" in which the block of data is transformed. We define this round transformation as \verb!Round(State,RoundKey)!, where State is the current 16 byte block selected out of the overall text and the RoundKey is a key derived from the input key using a key schedule as defined in \textsection\ref{subsubsec:aes_keys}. The number of rounds is dependent on the key size:
    
   
      \begin{center}
      \begin{tabular}{r|c|c|}
        \multicolumn{1}{r}{}
         &  \multicolumn{1}{c}{Key Length (bits)}
         & \multicolumn{1}{c}{Number of Rounds} \\
        \cline{2-3}
        AES-128 & 128 & 10 \\
        \cline{2-3}
        AES-192 & 192 & 12 \\
        \cline{2-3}
        AES-256 & 256 & 14 \\
        \cline{2-3}
      \end{tabular}
      \end{center}
      
    The \verb!Round! function consists of four functions:
    
    \begin{verbatim}
Round(State,RoundKey) {
  SubBytes(State)
  ShiftRows(State)
  MixColumns(State)
  AddRoundKey(State,RoundKey)
}
\end{verbatim}

    It should be noted that the final round, irrelevant of key size, is different in that it does not compute a \verb!MixColumns! transformation.
    
    \subsubsection{Transformations}
    
    The four internal functions within AES are known as \emph{transformations} because they modify the current block of data in some way. Each function is invertible, and as such we will only be describing the transformations for \emph{encryption} -- decryption is simply running the round in reverse.
    
    All functions work within a finite field. Addition is performed in GF($2$), which presents an easy implementation as bytes are represented in base-2 and so two bytes can be added together using XOR\cite{Gladman:2007aa}:
    
    \begin{center}
      \verb!01010111! $\oplus$ \verb!10000011! $\equiv$ \verb!11010100!
    \end{center}
    
    However, multiplication is more complicated. Given two elements within this field, each can have powers of $n^7$ meaning multiplication of the two will result in $n^{14}$ which is a value outside of the field (thus meaning it cannot be represented within a byte). To handle this, all polynomial multiplications are calculated modulo an irreducible polynomial $f(x)$ over the field GF($2^8$):
    
    \begin{center}
      $f(x) = x^8 + x^4 + x^3 + x + 1$
    \end{center}
    
    \paragraph{SubBytes}
    
    transforms the state block by replacing each byte value with a corresponding value in a substitution table known as an \emph{S-Box}. The S-Box is calculated by\cite{Standards:2001aa}:
    
    \begin{enumerate}
      \item Taking the multiplicative inverse in the finite field GF($2^8$).
      \item Apply an affine transformation:
        \begin{center}
          $b_i = b_i \oplus b_{(i+4) \mod 8} \oplus b_{(i+5) \mod 8} \oplus b_{(i+6) \mod 8} \oplus b_{(i+7) \mod 8} \oplus c_i$
        \end{center}
        
        where $b_i$ is the $i^{th}$ bit of the byte, and $c_i$ is the $i^{th}$ bit of the byte \verb!63!.
    \end{enumerate}
    
    Given the S-Box table consisting of 16 rows and 16 columns (0-9a-f), and a byte $b$, we determine the substitution from the table to be the $i^{th}$ row and $j^{th}$ column where $i$ is the leftmost four bits of the byte, and $j$ the rightmost four.
    
    The purpose of this transformation is to introduce non-linearity to what is effectively a substitution cipher. Without this, the cipher would be susceptible to a differential analysis attack, meaning an attacker can exploit the difference between two plaintext and ciphertext pairs.
    
    \paragraph{ShiftRows} is a simple transformation the cyclically shifts the last three rows of the state with given offsets. Given a byte $b_{i,j}$ in a matrix, it's new position after transformation is $b_{i,j} = b_{i,(c+i) \mod 4}, 0 \leq c < 4$. The first row is unaffected.
    
    \begin{center}
    $State = \begin{pmatrix}
      b_{0,0} & b_{0,1} & b_{0,2} & b_{0,3} \\
      b_{1,0} & b_{1,1} & b_{1,2} & b_{1,3} \\
      b_{2,0} & b_{2,1} & b_{2,2} & b_{2,3} \\
      b_{3,0} & b_{3,1} & b_{3,2} & b_{3,3} \\
    \end{pmatrix}$
    
    $State' = \begin{pmatrix}
      b_{0,0} & b_{0,1} & b_{0,2} & b_{0,3} \\
      b_{1,1} & b_{1,2} & b_{1,3} & b_{1,0} \\
      b_{2,2} & b_{2,3} & b_{2,0} & b_{2,1} \\
      b_{3,3} & b_{3,0} & b_{3,1} & b_{3,2} \\
    \end{pmatrix}$
    \end{center}
    
    \paragraph{MixColumns}
    
    handles each column in a state as 4-byte words, and considers them as polynomials over the finite field GF($2^8$) multiplied modulo $x^4 + 1$ with the polynomial $a(x) = \verb!03!x^3 + \verb!01!x^2 + \verb!01!x + \verb!02!$. $a(x)$ is represented as a matrix:
    
    \begin{center}
      $\begin{bmatrix}
        02 & 03 & 01 & 01 \\
        01 & 02 & 03 & 01 \\
        01 & 01 & 02 & 03 \\
        03 & 01 & 01 & 02
      \end{bmatrix}$
    \end{center}
    
    Given a column $c$, we get:
    
    \begin{center}
      $ \begin{bmatrix}
        b_{0,c} \\ b_{1,c} \\ b_{2,c} \\ b_{3,c}
      \end{bmatrix} =
      \begin{bmatrix}
        02 & 03 & 01 & 01 \\
        01 & 02 & 03 & 01 \\
        01 & 01 & 02 & 03 \\
        03 & 01 & 01 & 02
      \end{bmatrix}
      \begin{bmatrix}
        b_{0,c}^{'} \\ b_{1,c}^{'} \\ b_{2,c}^{'} \\ b_{3,c}^{'}
      \end{bmatrix}$
    \end{center}
    
    The purpose of \verb!MixColumns! along with \verb!ShiftRows! is to introduce a higher level of entropy in the message space where, due to the fundamentals of natural languages, low entropy distribution is highly likely.
    
    \paragraph{AddRoundKey} uses a bitwise XOR operation to add the current round key to the state. As with \verb!MixColumns!, the state is handled column-by-column in 4-byte words, which are XOR'd with the matching 4-byte words in a 16-byte round key block.
    \label{para:addroundkey}
    
    \begin{center}
      $\begin{bmatrix}
        b_{0,0} & b_{0,1} & b_{0,2} & b_{0,3} \\
        b_{1,1} & b_{1,2} & b_{1,3} & b_{1,0} \\
        b_{2,2} & b_{2,3} & b_{2,0} & b_{2,1} \\
        b_{3,3} & b_{3,0} & b_{3,1} & b_{3,2} \\
      \end{bmatrix} \oplus
      \begin{bmatrix}
        k_{0,0} & k_{0,1} & k_{0,2} & k_{0,3} \\
        k_{1,1} & k_{1,2} & k_{1,3} & k_{1,0} \\
        k_{2,2} & k_{2,3} & k_{2,0} & k_{2,1} \\
        k_{3,3} & k_{3,0} & k_{3,1} & k_{3,2} \\
      \end{bmatrix}$
    \end{center}
    
    Which, for a given column $c$, equates to $[b_{0,c},b_{1,c},b_{2,c},b_{3,c}] \oplus [k_{0,c},k_{1,c},k_{2,c},k_{3,c}]$.
    
    The purpose of this transformation is apparent: apply the secret to the message.
    
    \subsubsection{Keys}
    \label{subsubsec:aes_keys}
    
    Keys are input or generated as random $n$-bit byte arrays, where $n \in \{128,192,256\}$. This key is converted in to a \emph{key schedule} as defined by a key expansion method, which consists of $p$ key blocks (known as round keys) where $p$ is equal to the number of rounds for the given key size. When on the $p^{th}$ round of encryption, the $p^{th}$ key schedule block is added to the current state block (\textsection\ref{para:addroundkey}).
    
    Taking each column, as usual, in a block as a 4-byte word and starting with word $(i+4)$ where the first four words are the initial key, there are four steps to generating a word to become part of the key schedule:
    
    \begin{enumerate}
      \item \textbf{RotWord} -- similar to \verb!ShiftRows!, it cyclically rotates a word $[b_0,b_1,b_2,b_3]$ in to $[b_1,b_2,b_3,b_0]$.
      \item \textbf{SubWord} -- \verb!SubWord! replaces each byte with a corresponding value from the S-Box table, using the same logic as  \verb!SubBytes!.
      \item \textbf{XOR} -- the word is XOR'd with the $(i^{th}-4)$ word.
      \item \textbf{Rcon} -- the Rcon table is a \emph{round constant word array}, the matching columns of which are XOR'd with the current word.
    \end{enumerate}
    
    However, this only occurs for the first word in each 4 word block of the key schedule, and for 256-bit keys the \verb!RotWord! step is omitted completely. The remaining 3 words simply XOR the $(i^{th}-1)$ and $(i^{th}-4)$ blocks.
    
    An excellent animation that covers all aspects of the AES process can be found at \cite{Straubing:2005aa}.
  
  \subsection{Modes of Operation}
  
  Like most block ciphers, AES has a number of \emph{modes of operation}. The simplest of which is what we've been describing so far: divide the plaintext in to $n$-bit blocks (where the message length $m_l > n$ and transform each block. This is know as the electronic-codebook (ECB). There are four most-prevalent modes: ECB, CBC (Cipher Block Chaining), CFB (Cipher Feedback), and OFB (Output Feedback).
  
  \paragraph{ECB} mode is as above:
    
  \begin{center}
    Where $C$ is a ciphertext block, $P$ and plaintext and $E$ the encryption function, 
    $C_i = E_k(P_i)$, $P_i = D_k(C_i)$
  \end{center}
  
  Blocks are enciphered independently of all other blocks, and so any errors in a block do not \emph{propagate} throughout the rest of the ciphertext, meaning the majority of the ciphertext will still be able to be decrypted.
  
  However, this independence of blocks means that malicious blocks can be substituted into a ciphertext. Alongside this, we are open to frequency analysis and other ciphertext-only attacks. ECB is generally not recommended for use in a production environment.
  
  \paragraph{CBC} mode involves XOR'ing the first plaintext block with a random bit string known as the initialisation vector (IV) and then for each block the plaintext is XOR'ed with the previous block.
  
  \begin{center}
    $C_0 = IV$ and for $i$th block $C_i = E_k(P_i \oplus C_{i-1})$, $P_i = D_k(C_i) \oplus C_{i-1}$
  \end{center}
  
  CBC is dependent on the correct ordering of the block chain, consequently rearranging the blocks or modifying any will affect the output of decryption. While this is beneficial in preventing attacks, it has an affect on error propagation. As a block $c_i$ is dependent on $c_{i-1}$, if $c_{i-1}$ contains any error, the decryption of block $c_i$ will also be affected, which also opens the algorithm up to attacks by altering the bits of $c_{i-1}$. However blocks $c_{i+2}$ are not affected by errors in $c_i$, providing a form of error recovery.
  
  \paragraph{CFB} mode is similar to CBC in that it is effectively the reverse of the CBC operation and also making use of an initialisation vector.
  
  \begin{center}
    $C_0 = IV$, $C_i = E_k(C_{i-1}) \oplus P_i$, $P_i = E_k(C_{i-1} \oplus C_i$
  \end{center}
  
  \paragraph{OFB,} interestingly, turns the block cipher effectively in to a stream cipher.   
  
  \begin{center}
    $C_i = P_i \oplus O_i, P_i = C_i \oplus O_i$, \\
    where $O_i = E_k(I_i)$ and $I_0 = IV$, $I_i = O_{i-1}$.
  \end{center}
  
  OFB excels over other modes with regards to the avoidance of error propagation and can recover from bit errors in blocks. Conversely, the \emph{loss} of block bits results in the keystream alignment being damaged meaning decryption is not possible.
  
  We will be implementing ECB, as it is clearer to explain, with a discussion on how to implement other modes. And because the purpose of this is not to develop a perfectly secure algorithm. The underlying algorithm is not affected by any of these modes, as each block is still encrypted and decrypted using the transformations listed in \textsection\ref{subsec:aes_algo}. This means that implementing these modes as either an option or a permanent modification is relatively easy in terms of development time.
  
  \subsection{Implementation}
  
  As with the public-key algorithms, the AES algorithm will be realised using Java so that it can be referenced later by the Enigma application.
  
    \subsubsection{A Note on Block Representation}
    
    As is apparently obvious and shown in the algorithm description, blocks are 4 by 4 arrays of bytes. However, copying parts of blocks (represented as Java \verb!byte[]!) and running calculations on them is awkward to visualise as the blocks will have to be made up of multi-dimensional arrays. As such, throughout the implementation we will use both a multi-dimensional array block representation and a one-dimensional array representation, the difference of which is made clear wherever each is used. \textsection\ref{subsubsec:aes_commonutils} defines two methods that will convert between the two representations.
    
    \subsubsection{Key Generation}
    
    Key generation is simple as keys are just random bit strings of a length equal to the desired key size. We have created an \verb!enum! called \verb!KeySize! (the idea of which is partially from a project identified by the package \emph{watne.seis720.project}) which takes the key size as input to a constructor, and provides helper methods such as \verb!getNumberOfRounds()! to provide a persistent method of retrieving the current requirements for the key size without having to have hard-coded values within transformations and methods.
    
    Keys are represented as an object \verb!Key! with constructor \verb!Key(KeySize k)! meaning it is provided with a \verb!KeySize! enum object which defines the size of the key to be generated. \\
    
    \begin{lstlisting}
public Key(KeySize k) throws DataFormatException {
  byte[] key = new byte[k.getKeySizeBytes()];
  this.ksize = k;
  
  SecureRandom rng = new SecureRandom();
  rng.nextBytes(key);
}
\end{lstlisting}
    
    This is perhaps misleading as random number generators tend to take a seed and a length and return a random number. However in this case we provide the generator with a byte array the size of the key needed and it fills it with random data.
    
    As the \verb!Key! object represents a key within a session, it also provides a method to return the expanded version of the key, however this will be covered in \textsection\ref{subsubsec:aes_transformations}.
    
    \subsubsection{Constants}
    
    The S-Boxes used by the \verb!SubBytes! transformation can be computed on-the-fly, however given that the values are always the same and independent of any plaintext or ciphertext input, it makes little sense to do so. As such, we use pre-computed S-Boxes that are stored in a class \verb!AES_Constants! as a multi-dimensional array.
    
    Alongside the the S-Boxes, \verb!AES_Constants! also stores the matrices used for the \verb!MixColumn! and \verb!InvMixColumn! transformation as they are also fixed, rendering calculation of them on-access redundant. 
    
    \subsubsection{Common Utilities}
    \label{subsubsec:aes_commonutils}
    
    \paragraph{State Representation} varies between methods. As we said previously, occasionally it is more useful to work on blocks as one-dimensional arrays rather than multi-dimensional, and as such we will need utilities that convert between the two types: \verb!arrayTo4xArray()! and \verb!array4xToArray()!. They are both reasonably simple -- the latter loops through each of the 4 rows of bytes and appends them to a 16-byte array using \verb!System.arraycopy()!: \\
    
    \begin{lstlisting}
for (int i=0;i<4;i++) {
  System.arraycopy(array[i],0,array1x,(i*4),4);
}
\end{lstlisting}

    The former does the same, but in reverse, copying each section of 4 bytes from a 16-byte array in to the matching row in a multi-dimensional array: \\
    
    \begin{lstlisting}
for (int i=0;i<4;i++) {
  System.arraycopy(array,(i*4),array4x[i],0,4);
}
\end{lstlisting}
    
    \paragraph{Padding} is required for messages that are not divisible by the block size, 16. Adding zero bytes to ``fill out" a byte array will work, however after decryption how will we know how many padding bytes to remove? \cite{Kaliski:2000aa}, section 6.1.1, defines the padding string as consisting of $8-(||M|| \mod 8)$ bytes with the value $8-(||M|| \mod 8)$ for a message $M$. For example, for a message of length 6, we get:
    
    \begin{center}
      $M' = M \ || \ 0202$
    \end{center}
    
    As the padding bytes are each equal the number of padding bytes used, we know how many to remove. However, in our case we are using block lengths of 16 and so our padding strings will consist of $16-(||M|| \mod 16)$ bytes equalling $16-(||M|| \mod 16)$.
    
    \paragraph{Multiplication Over the Finite Field} could be completed arithmetically, however it is far simpler and more efficient to convert it in to using bitwise arithmetic. For example, the irreducible polynomial $m(x) = x^8 + x^4 + x^3 + x + 1$ can be represented as the byte \verb!0x11b!, and thus addition over the finite field can be calculated through XOR'ing the byte with \verb!0x11b!. \\
    
    \textcolor{red}{Add more from gladman}
    
    \begin{lstlisting}
public static byte FFMul(byte a, byte b) {
  byte r = 0;
  
  while (a != 0) {
    if ((a & 1) != 0) r = (byte)(r ^ b);
    
    // Repeatedly multiply by (1)
    b = (byte)(b << 1);
    
    // If the result is of degree 8
    // add m(x)
    if ((byte)(b & 0x80) != 0)
        b = (byte)(b ^ 0x1b);
    
    a = (byte)((a & 0xff) >> 1);
  }
      
    return r;
}
\end{lstlisting}
  
    \subsubsection{Transformations}
    \label{subsubsec:aes_transformations}
    
    \textsection\ref{subsec:aes_algo} explicitly defines the transformation algorithms used here. Refer to that section for detailed, non-programmatic descriptions.
    
    \paragraph{SubBytes}
    
    We have already decided that S-Box values are to be pre-computed and stored statically, so retrieving the S-Box value for each byte in a bock (the purpose of \verb!SubBytes!) is trivial. \\
    
    \begin{lstlisting}
for (int i=0;i<block.length;i++) {
  for (int j=0;j<4;j++) {
    block[i][j] = AES_Transformations.getSBoxValue(block[i][j]);
  }
}
\end{lstlisting}

    Here we are looping through each row and then column of a block and retrieving the S-Box value using a helper function: \\
    
    \begin{lstlisting}
public static byte getSBoxValue(byte o) {
  // Get 4 left-most and right-most bits
  int i = ((o & 0xf0) >> 4);
  int j = (o & 0x0f);

  return AES_Constants.SBOX[i][j];
}
\end{lstlisting}
    
    \paragraph{ShiftRows}
    
    cyclically shifts the byte in the last three rows of a state block. With starting row $r = 0$ in the set ${0,1,2,3}$, each row is shift $r$ times. \\
    
    \begin{lstlisting}
public static byte[][] shiftRows(byte[][] state) {
  byte[][] new_state = new byte[4][4];
		
  // Keep first row
  new_state[0] = state[0];
		
  for (int i=1;i<=3;i++) {
    // Copy r columns to the end of the new state
    System.arraycopy(state[i],i,new_state[i],0,4-i);
    // Copy the 4-r columns to the start of the new state
    System.arraycopy(state[i],0,new_state[i],4-i,i);
  }
		
  return new_state;
}
\end{lstlisting}
    
    \paragraph{MixColumns}
    
    iterates through blocks column by column and considers each column as a four variable ploynomial. As such we multiply each byte in a column with the predefined \verb!MixColumn! matrix, and then XOR all the multiplication results together.  \\
    
    \begin{lstlisting}
for (int col=0;col<4;col++) {
  column[col] = AES_Utils.FFMul(AES_Constants.MIXCOL[col][0], state[0][i]) ^
    AES_Utils.FFMul(AES_Constants.MIXCOL[col][1], state[1][i]) ^
    AES_Utils.FFMul(AES_Constants.MIXCOL[col][2], state[2][i]) ^
    AES_Utils.FFMul(AES_Constants.MIXCOL[col][3], state[3][i]);				
}
\end{lstlisting}
    
    \paragraph{AddRoundKey}
    
    uses an XOR operation to add the round key to the current state block. For a given row $r$, we iterate through the bytes in the expanded key (\textsection\ref{para:aes_keyexp}) with indexes in the interval $[16r,(16r)+16]$ and XOR them in order with each byte in the block. \\
    
    \begin{lstlisting}
byte[] exp_key = key.getExpandedKey();
		
// Initial round starts from index 0
// All further rounds start from 16 bits * round, i.e.
// 16 bytes ahead of the last round
int index = r*16; 
		
for (int col=0;col<4;col++) {
  for (int row=0;row<4;row++) {
    state[row][col] = (byte)(block[row][col]^exp_key[index++]);
  }
}
\end{lstlisting}
    
    \paragraph{KeyExpansion}
    \label{para:aes_keyexp}
    
    is the most complex of the transformations, and is actually part of the \verb!Key! class. The algorithm itself is defined fully in \textsection\ref{subsec:aes_algo} However, as shown in \cite{Wagner:2003ly}, this can be done in one encapsulated method. Following along the same lines, we use single bytes rather than 4-byte words as defined in \cite{Standards:2001aa}. Our implementation of key expansion is a simplified (and more readable) version of the method found in \cite{Wagner:2003ly}. \\
    
\begin{lstlisting}
// Make following FIPS-197 pseudo-code easier and use their conventions
int Nk = this.getKeySize().getKeySizeWords();
int Nr = this.getKeySize().getNumberOfRounds();
int Nb = 4;

this.expanded_key = new byte[Nb*4*(Nr+1)];

byte[] cur_word = new byte[4];
  
int i;  
for(int j=4*Nk; j < 4*Nb*(Nr+1); j+=4) {
   i = j/4;
   
   // Get the next 4 bytes (word) of the key
   for (int k=0;k<4;k++) cur_word[k] = this.expanded_key[j-4+k];
   
   if (i % Nk == 0) {
     // Loop through the current word
     for (int k=0;k<4;k++) {
       // Determine which byte of the word to use
        byte temp = cur_word[(k==3) ? 0 : k+1];
        // Determine rcon value
        byte rcon = (k == 0) ? AES_Constants.RCON[(i/Nk)-1] : 0;
        // xor each byte in the word with temp value and rcon value
        cur_word[k] = (byte)(AES_Transformations.getSBoxValue(temp) ^ rcon);
     }
   // As defined in FIPS-197...
   // For 256-bit keys we apply SubWord() to
   // expanded_key[i-1] before xoring below
   } else if ((Nk==8) && ((i%Nk)==4)) {
     for (int k=0;k<4;k++)
       cur_word[k] = AES_Transformations.getSBoxValue(cur_word[k]);
   }
   
   // The actual work..
   // xor each byte of the current word with the
   // matching byte in the 4 words behind
   for (int k=0;k<4;k++) {
     this.expanded_key[j+k] = (byte)(this.expanded_key[j - 4*Nk + k]
                              ^ cur_word[k]);
   }
}
\end{lstlisting}
    
    \subsubsection{Algorithm}
    
    Now that the transformations and common utilities have been defined, it is trivial to implement the actual algorithm that executes the rounds on each block.
    
    We define a method \verb!encrypt()! that given a byte array will return an enciphered byte array by looping through each block of the given plaintext and enciphering it before placing it back in to a final ciphertext byte array at the relevant array index. The work is done by a method \verb!cipher()!: \\
    
    \begin{lstlisting}
private byte[] cipher(byte[] block) throws DataFormatException {
  // Conver the given 1x16 byte array in to a 4x4 array that
  // can be used by the transformation functions
  byte[][] state = AES_Utils.arrayTo4xArray(block);
  
  state = AES_Transformations.addRoundKey(state, this.getKey(), 0);
  
  // Iterate over the block for the number of rounds defined by the 
  // type of key
  for (int r=1;r<this.getKey().getKeySize().getNumberOfRounds();r++) {
    state = AES_Transformations.subBytes(state);
    state = AES_Transformations.shiftRows(state);
    state = AES_Transformations.mixColumns(state);
    state = AES_Transformations.addRoundKey(state, this.getKey(), r);
  }
  
  // The final round excludes the mix columns transformation
  state = AES_Transformations.subBytes(state);
  state = AES_Transformations.shiftRows(state);
  state = AES_Transformations.addRoundKey(state, this.getKey(),
                                                 this.getKey()
                                                    .getKeySize()
                                                    .getNumberOfRounds());
  
  return AES_Utils.array4xToArray(state);
}
\end{lstlisting}

  \verb!decrypt()! and \verb!invcipher()! work identically, except \verb!invcipher()! carries out the inverse transformations defined in \textsection\ref{subsubsec:aes_transformations}.
  
  These methods are contained within an object \verb!AES!, which upon instantiation requires the setting of a plain- or cipher-text and a key.
  \subsection{Summary}
  
  As we can see, AES is relatively simple to implement in an efficient way.  

% Chapter 5

\chapter{Identification and Authentication} 
\label{Chapter5}
\lhead{Chapter 5. \emph{Authentication}} 

\section{Overview and Intentions}

\section{Basic and Common Schemes}

\section{Digital Signatures}

\section{Certificates}

\section{Other Methods}

% Zero knowledge, blind signatures, etc 

% Chapter 6

\chapter{Enigma: A Testbed} 
\label{Chapter6}
\lhead{Chapter 6. \emph{Enigma: A Testbed}} 

\section{Overview and Intentions}

The intention of this project is to produce an application that allows two users to securely identify and authenticate one another, and communicate with text messages via a presumed-insecure network. However, the primary \emph{goal} is to create a testbed that allows any cryptographic algorithms to be integrated in place, to provide a platform for further analysis, such as comparison of run-times between open and closed source implementations.

\section{Engineering Methodologies and Planning}

\subsection{Methodology}

The main focus of the development process was to break the application down in to its component pieces, and iteratively develop them so that minimalist functionality sets could be produced and then combined in the shortest possible times. The overarching category for this style of production is known as \emph{agile development}: a practice based around iterative and incremental development. More specifically, a subset of agile development will be used -- Scrum. The goals of Scrum fit neatly with the desired development cycle in that programming is done in "sprints," with a deadline organised at the start of these sprints and a set of tasks that must be completed by the end. A sprint can last anywhere between one week and one month. This, combined with test driven development (TDD, writing tests for functionality before producing the functionality), produces a set of components matching a well-defined requirements document that can be combined in to the final product.

However, without results, methodologies are meaningless. One should not concentrate too heavily on explicitly following a development process, particularly when working alone, otherwise work output can be reduced significantly by the overhead and "red tape." Because of this, the methodology of development will not be encountered again in this document, however it is of interest to know the basics as above.

\subsection{Documentation}

Documentation is an extremely important part of software engineering. It is a broad concept, and encompasses: requirements and specification, design overview, technical details, user manuals and even marketing information. However, in terms of the software development process, it is only the first three that are of direct importance. 

It is said a program listing should be documentation unto itself: the programming style should allow an overall structure and purpose to be easily determined from examining the code \cite{McConnell:2004tv}. However, this is an idealistic view and design and technical documentation are of great importance, not only for the developer currently producing the software, but for those possible developers in the future who have to maintain the code.

As a major component of this project is to develop a library of cryptographic algorithms, having a clear and accurate listing of all available methods in the API is vital. Conveniently, Java and the Eclipse IDE are tightly integrated with \emph{JavaDoc}\footnote{http://java.sun.com/j2se/javadoc}, a documentation generator that automatically produces standardised API documentation in HTML format based on comments inserted in the code. The comment blocks -- distinguished using the format \verb!/** ... */! -- contain a method description, and a number of control sequences that give detailed information about what the method returns, the parameters it takes and other details such as exceptions thrown.

JavaDoc outputs are bundled with the appropriate packages in the file tree, and manual technical documentation is found in the appendices. A requirements specification has been compiled in \emph{\textsection \ref{AppendixA}}.

\section{Application Development}

  \subsection{User Interface}
  \emph{Packages covered in this section: com.cyanoryx.uni.enigma.gui.*}
  
    \subsubsection{GUI Frameworks}
    
      \paragraph{SWT} -- the Standard Widget Toolkit -- initially created by IBM and now maintained by the Eclipse Foundation, it's a modern alternative to Swing. Initially SWT was used (predominantly in tests) for the Enigma application, however as development continued it became more apparent that it was introducing a level of complexity to the interface code so much so that development slowed to an unmanageable speed. SWT benefits from its use of native component libraries, generally resulting in a more congruent user experience, however it falls down when it comes to ease of development.
      
      \textbf{Pros and cons of Swing}
      
       \begin{center}
       \begin{tabular}{ | p{6cm} | p{6cm} |}
          \hline
          Pros & Cons \\ \hline \hline
          Part of the JDK, meaning there's no need for native system libraries & The look and feel does not always match well with the native system \\ \hline
          No differences in development between platforms and systems & \\ \hline
          Very good documentation available, particularly from Sun & \\ \hline
          Mature and well supported & \\ \hline
          Is supported by official Java extensions like OpenGL & \\
          \hline
        \end{tabular}
      \end{center}
      
      \textbf{Pros and cons of SWT}
      
      \begin{center}
       \begin{tabular}{ | p{6cm} | p{6cm} |}
          \hline
          Pros & Cons \\ \hline \hline
          Uses native components & Native libraries must be available for all supported systems \\ \hline
          Also supported by the Eclipse editor & Some native resources are not available on other systems, and so portability can be damaged \\ \hline
          Documentation is good (though not as good as Swing) & Requires manual management of resources rather than, for example, SWT disposing windows, they must be done by the developer. \\ \hline
          SWT programs can also integrate Swing components & SWT requires separate libraries to be distributed for 32- and 64-bit systems. \\ \hline
          SWT is supposedly faster at rendering than Swing, though only minimally & \\
          \hline
        \end{tabular}
      \end{center}
      
      SWT appears initially simpler to use thanks to its use of the Model--View--Controller design pattern, the ability to "plug in" different look and feel settings, and so on, however it introduces the necessity to manage resources manually, rather than following the Java standard method of automatic resource disposal (known as the garbage collector).
      
      Swing was selected as the framework to be used, primarily due to its current prevalence over SWT. Purely comparing speed statistics, SWT comes out on top, however the community behind Swing is far larger and more important than the downsides introduced by using Swing, and clearly the balance of pros and cons is in Swing's favour.
      
    \subsubsection{Designs}
    
    Being an application purely for testing purposes, the interface can have a certain leniency in terms of usability as those utilising it will likely have a good understanding of systems anyway. However, more importantly, the interface should be "invisible," meaning that the user should not have to think about using it and can focus on the task they are trying to complete -- e.g. comparing encryption algorithms.
    
    The application will have four main windows:
    
    \begin{enumerate}
      \item Connect Window -- The main starting point of the application. Users will enter the IP address of another Enigma server, or select an address from a "recently connected" menu.
      \item Chat Window -- Where the actual communications will take place. It will consist of a main listing box to display a message history for the current conversation, and have an input box for sending messages. It will also contain options for: Regenerating session keys, changing the cipher type, changing the agreement cipher type, and viewing the log window for this session.
      \item Log Window -- Can be used to log info, warning and error messages that do not cause a general program crash.
      \item Preferences -- Used to modify basic preferences, like default ciphers, username, etc.
    \end{enumerate}
    
     \begin{figure}
       \centering
       \includegraphics[scale=0.7]{./Figures/Ch6/6-3-1-2a.pdf}
       \caption{Overall Application Flow}
       \label{fig:app_flow}
     \end{figure}
    
    Figure \ref{fig:app_flow} shows a basic overview of how the application will be used.
    
    As developed with Swing, the four main windows look as so (on Mac OS X):
    
    \subsubsection{Connect Window}
    
    \begin{figure}
      \centering
      \includegraphics[scale=0.7]{./Figures/Ch6/6-3-1-2-connect_window.pdf}
      \label{fig:connect_window}
    \end{figure}
    
    The Connect window comprises of two main pieces of logic:
    
    \begin{enumerate}
      \item Server creation.
      \item Connection making.
    \end{enumerate}
    
    Firstly, the Connect window is the main focal point of the application and as such the creation of a \verb!Connect! object signifies the start of the Enigma application. \verb!Connect#Connect()! creates a new \verb!Server! object with a random port number and initiates the process of creating the UI for the window.
    
    As with most Swing-utilising classes, event handlers are used to process events triggered by user input -- such as a button press -- and in this case the main event handler will be that of the connect button, as seen in \ref{fig:connect_window}. When a valid IP address is input and the button is clicked, the \verb!connect()! method is run:
    
    \begin{lstlisting}
private void connect(String address, String remote_port) {
  try {
    // Display a loading screen
    Connect.this.showLoading();
    
    // As we are the initialiser,
    // generate a session ID
    String id = ""+(new Random().nextInt(100));
    
    // Create a new Session to store data for the remote server
    Session session = Server.createClient(address,
                      remote_port,
                      ""+port,
                      new User("remote user"),
                      id);
                      
    // Store the Session in the server's session index
    Connect.this.server.getSessionIndex().addSession(session);

    // Send our desired agreement method
    session.sendAuth("method",
                     "agreement",
                     new AppPrefs().getPrefs()
                                   .get("default_asym_cipher","RSA"),
                                        id);
                                        
    // Send our public key and certificate
    session.sendAuth("cert",
                     "agreement",
                     Base64.encodeBytes(new Certificate(new File("./cert"))
                                            .toString()
                                            .getBytes()),
                     id);
    
    // Store the remote server's IP address
    new AppPrefs().getPrefs().put("last_connections",
                   address+":"+port+";"+new AppPrefs().getPrefs()
                                                      .get("last_connections",""));
  } catch (Exception e) {
    [...] // Display an error
  } finally {
    try {
      // Always reinstate the UI
      Connect.this.recreateUI();
    } catch (ParseException e) {
      e.printStackTrace();
      System.exit(1);
    }
  }
}
    \end{lstlisting}
    
    Here we make use of a method \verb!Server#createClient()!, which is a bit of a misnomer based around a legacy concept within the application where a local user would run both a server and a "client," for sending and receiving data, respectively. This was designed as such due to some restrictions on the use of sockets (see \textsection\ref{sec:probs}), however this was removed for simplicity and due to the confusion introduced by running a "client" locally.
    
    \verb!Server#createClient()! is a simple, static method that creates a \verb!Session! object with the supplied attributes and returns it, along with a \verb!Conversation! object for the chat window.
  
    \subsubsection{Chat Window}
    \label{subsubsec:chat}
    
    \begin{figure}
      \centering
      \includegraphics[scale=0.7]{./Figures/Ch6/6-3-1-2-chat_window.pdf}
      \label{fig:chat_window}
      \caption{Chat Window}
     \end{figure}
    
    The conversation has three tasks:
    
    \begin{enumerate}
      \item Display received messages.
      \item Take the user's input and send the message to the relevant \verb!Session! object for encryption and transmission, and then display that message in the window (if sent).
      \item Provide access to the following functionality:
        \begin{enumerate}
          \item Opening/closing the log window.
          \item Regenerating the session keys.
          \item Viewing the remote user's certificate.
          \item Toggling encryption on and off.
        \end{enumerate}
    \end{enumerate}
    
    Each are relatively simple as most of the hard work is handed off to other objects within the \verb!Server!'s jurisdiction. 
    
    \paragraph{The display of messages} is implemented using a \verb!JTextPane! which, unlike a \verb!JTextArea!, honours the use of font styling, colours and post-creation programmatic string insertion. The \verb!updateMessage()! method takes a user name and a message:
    
    \begin{lstlisting}
public synchronized void updateMessage(String name, String message) throws BadLocationException {
  StyledDocument d = messages.getStyledDocument();
      
      SimpleAttributeSet kw = new SimpleAttributeSet();
      StyleConstants.setBold(kw,true);
      
  d.insertString(d.getLength(),name+": ",kw);
  d.insertString(d.getLength(), message+"\n", new SimpleAttributeSet()); // Use a blank attribute set for the actual message text
}
    \end{lstlisting}
    
    Two new Java classes are required: \verb!StyledDocument! and \verb!SimpleAttributeSet!. The former converts the messages \verb!JTextPane! in to an object that can be formatted, and the latter is a set of attributes that can be applied as formatting to the text we are appending to the text pane. 
    
    \verb!Conversation! windows are somewhat different from other interfaces in that they are persistent to a connection -- conversations windows are stored in \verb!Session! objects (as we will see later). Because of this, when a message is received, the handler retrieves the \verb!Conversation! object and calls \verb!updateMessage()!.
    
    \paragraph{Sending messages} is handled entirely by \verb!Session! objects, and so when a user enters text and hits enter or clicks "send," the \verb!Session#sendMessage()! method is called.
    
    Displaying the log window, viewing the remote user's certificate and regenerating the session keys are all similarly simplistic in their implementation. The latter utilises the \verb!Session#sendAuth()! method to regenerate a key and send it, however this is discussed in detail later. The other two buttons simply toggle the current display status of the two windows.
    
    Toggling the encryption status is slightly more complex, however. The current status of encryption for the conversation must be checked, and turning it on or off depending on the result, also checking if the user allows for unencrypted conversations.
    
    \subsubsection{Log Window}
    
    \begin{figure}
      \centering
      \includegraphics[scale=0.5]{./Figures/Ch6/6-3-1-2-log_window.pdf}
      \label{fig:log_window}
      \caption{Log Viewer}
     \end{figure}
    
    The log window itself is a simple setup (\verb!LogWindow!): a Swing \verb!JFrame! wrapper with a disabled \verb!JTextArea! and a method for handling input to be appended to the text area. The complexity behind it, however, is in the \verb!LogHandler! class. \verb!LogHandler! extends \verb!java.util.logging.Handler!, but why not just have a \verb!LogWindow! object and have each class manually update the text area? Using a \verb!Handler! implementation is all about \emph{extensibility}: it provides built-in capability for handling logging level settings (e.g. only log warnings, information, errors, etc.), along with applying formatters and filters. In the current version of Enigma, only the \verb!Conversation! class needs to log information and so the log handler does little data modification and simply passes the messages on to a \verb!LogWindow! object.
    
    For the full code behind the logging system, see \emph{com.cyanoryx.uni.enigma.gui.Log*.java}.
    
    \subsubsection{Preferences Window}
    
    User preference changed are handled by a single, unified preferences window, created by a \verb!com.cyanoryx.uni.enigma.gui.Preferences! class (which should be distinguished from \verb!Preferences! in the \emph{java.util.prefs} namespace). It is another simple class that implements a \verb!JFrame! containing multiple \verb!JTabbedPane! elements separating out preferences in to their relevant categories. 
    
     \begin{figure}
      \centering
      \includegraphics[scale=0.7]{./Figures/Ch6/6-3-1-2-preferences_window.pdf}
      \label{fig:pref_window}
      \caption{Preferences}
     \end{figure}
    
    Upon loading, each preference component is created with its value equal to the currently stored value, or if there is none, a default value. Each preference is governed by an event listener appropriate to the data type of the option:
    
    \begin{enumerate}
      \item \verb!ItemListener! -- boolean preferences use \verb!ItemListener#itemStateChanged()! to register for a checkbox state change.
      \item \verb!ActionListener! -- preferences with multiple choices implemented as drop down combo boxes register with \verb!ActionListener#actionPerformed()! to monitor item selection.
      \item \verb!DocumentListener! -- string based preferences, such as user name, are monitored using the update registering methods in \verb!DocumentListener! to detect update, deleted or insert text.
    \end{enumerate}
    
    When an event is generated, the updated preference is stored, meaning that the user never has to press any "save" or "apply" buttons.
    
    In the future, it would perhaps be interesting to retrieve the list of available settings and their datatypes, and automatically generate a preference interface based on this. However, this may pose problems for preferences with multiple choices (like in drop down boxes), as all the options will not be saved in the preferences storage.
    
    See \textsection\ref{subsec:prefs} for the actual implementation of preference storage.
    
    Each window was developed using the Swing framework, as mentioned, but the "look and feel" of the components were designated to match the JDK's interpretation of the system's theme, meaning that the components will be very similar to those used in native applications thus making the application more portable, at least in terms of appearance. 
    
  \subsection{Preferences}
  \label{subsec:prefs}
  
  It is necessary to have some persistent data between application sessions, for example the user's display name and their cipher preferences. There are many simple and many complex methodologies to solve this problem, the simplest seemingly being saving preferences to a text file with a custom format. However, as with any proprietary format it will be non-portable from the outset, and likely difficult to maintain in the future. Large amounts of "boiler--plate" code will be required to manage aspects like text-encoding, storage location, file formats, and so on which will unnecessarily add to development time and likely create bugs.
  
  The JDK comes with a class \emph{com.util.prefs.Preferences} for storing a collection of preference data. \verb!Preferences! stores data in a key-value persistent backing store that is independent of the system implementation. Given a "node" name, \verb!Preferences! will load data from the appropriate storage implementation, depending on the operating system. For example, it may store preferences in flat files, registries or even databases. This has one primary benefit: the developer does not need to concern themselves with the details of storage, but merely implement an application that creates a \verb!Preferences! object given a node name and \verb!Preferences! will handle the details.
  
  More specifically, \verb!Preferences! has a static method \emph{userNodeForPackage(String name)} which returns a \verb!Preferences! object containing handles that reference the persistent backing store. This \verb!Preferences! objects provides helper methods such as \emph{get(String key, String default\_val)} that will programmatically return the value for the key \verb!key!, handling any errors and using the default value if necessary. 
  
  However, this introduce an integrity and consistency issue: what if the node name changes? Do we store the node name in a centralised location? (introducing something of a chicken and egg problem). Or perhaps assume that it will never need to change? In our case, a wrapper class was created that returned a \verb!Preferences! object with a node name equal to \verb!getClass()!. This way the node will always be named after the package identifier for the given class, which can be refactored if necessary. \\
  
  \lstinputlisting{../../sys/EnigmaApp/src/com/cyanoryx/uni/enigma/utils/AppPrefs.java}
  
  \emph{This code can be found in com.cyanoryx.uni.enigma.utils.AppPrefs}
  
  We have also written a helper method that returns the last 10 IP connections, which are stored upon connection as a concatenated \verb!String! array.
  
    \subsubsection{Usage}
    
    \begin{lstlisting}
      Preferences p = new AppPrefs().getPrefs();
      p.get("preference_name","test");
    \end{lstlisting}
  
  \subsection{Internationalisation}
  
  Internationalisation, colloquially shortened to \emph{i18n} as 18 is the number of characters between the first and last letters, of an application is adding the possibility of easily creating new language sets that allow components and text in the application to be translated and shown in another language while retaining the ability to switch back to other languages. The purpose is to remove the need for software changes, and introduce a preference or setting that points the application to appropriate language set.
  
  In this case, the language sets are stored in \verb!.properties! files:
  
  \lstinputlisting{../../sys/EnigmaApp/bin/Enigma.properties}
  
  As with the preferences, a helper class is necessary to access the locale information and abstract some of the more verbose syntax needed for getting language information. In this case, a class \verb!Strings! has been created that on initialisation gets the locale preference and attempts to load the corresponding language set (known in \verb!Locale! parlance as a bundle), and provides a translate method that takes a string identifier, and returns the matching language string from the current locale bundle. \\
  
  \lstinputlisting{../../sys/EnigmaApp/src/com/cyanoryx/uni/enigma/utils/Strings.java}
  
  \emph{This code can be found in com.cyanoryx.uni.enigma.utils.Strings}
  
  \subsection{Drawable Graphics}
  
  Some elements of the user interface require the inclusion of local images and icons. For example, in the chat window the menu bar buttons contain representative icons rather than text. However, as with strings of text, hard-coded references to resources generally results in excessive maintenance if the resource location changes. As a solution to this, we will create a lightweight class modelled around the \emph{Android SDK} concept of "drawables." Drawables are graphic resources stored in the system, represented by an identifier which can be retrieved using the \verb!Drawable! class.\\
  
  \lstinputlisting{../../sys/EnigmaApp/src/com/cyanoryx/uni/enigma/utils/Drawable.java}
  
  \emph{This code can be found in com.cyanoryx.uni.enigma.utils.Drawable}
  
  As an example: if we decide that a database is more suitable for storing graphics than local files, the \verb!loadImage()! method can be modified to search the database for the identifiers, and all the (possibly) hundreds of references to images within other parts of the codebase will not have to change.

\section{Protocol Implementation}
  \emph{Packages covered in this section: com.cyanoryx.uni.enigma.net.*}
  
  \subsection{Overview}
  
  The purpose of the Enigma application is to provide an example application to be used for testing cryptographic algorithms, however the \emph{function} of the application is to provide a means of communication between two entities who are capable of running the same software. To do this we will be implementing a custom variation of the Jabber/XMPP protocols within a server that will be able to send and receive  XML data streams.
  
  Servers are simple to implement, particularly with Java's excellent networking support -- simply open a port locally that accepts connections, and parse any input and return outputs where necessary. However, there are many other aspects of running a server that are not covered by this in any form: for example, handling multiple connections efficiently and reliably, maintaining message integrity, and so forth. Because of this, we will be creating a server that is extensible and advanced in how it deals with connections and input. Alongside this, by introducing XML parsing capabilities and a well-defined protocol we are able to improve reliability (by being able to detect malformed XML) and make handling different types of input easier (by have distinct sets of XML tags for specific purposes).
  
    \subsubsection{Goals and Objectives}
    
    \begin{enumerate}
      \item Simplicity -- the server should have components which are only strictly necessary. Very little time should be spent on input processing.
      \item Easy to integrate with cryptographic algorithms -- this is harder to quantify, however it should result in software where there are only a small handful of points where algorithms will need to be connected, reducing the amount of repetition.
      \item Extensible -- it should be easy to add new handlers in the event the protocol is modified.
    \end{enumerate}
    
    For a full requirements specification, see \textsection \ref{AppendixA}.
    
  \subsection{Server Design}
  
    \subsubsection{Server}
    
      \begin{figure}
        \centering
        \includegraphics[scale=0.7]{./Figures/Ch6/6-4-2.pdf}
        \caption{A simplified overview of two servers communicating.}
        \label{fig:server_overview}
      \end{figure}
      
    The server is be split in to four major components:
    
    \begin{enumerate}
      \item Sessions -- connections between two entities will be managed by \verb!Session! objects stored in an index.
      \item Packets -- packets represent sections of an XML document are stored in a queue upon arrival, and are handled in a first come first served basis by the XML parser.
      \item XML parser -- the XML parser takes packets and converts them in to Java objects for use in handler endpoints.
      \item Handlers -- after a packet has been parsed, it is sent to the registered handler for its type, which processes the packet. For example, a message handler would only take \verb!<message>! packets and would send them to the correct GUI window for the sending entity.
    \end{enumerate}
    
    Various threads will be created to allow for concurrent processing of multiple connections, however all these threads will run under a parent \verb!Server! thread that maintains connections and sessions, along with dealing with registering packet handlers.
    
    The server uses \verb!java.net.ServerSocket! to create a socket that accepts all connections on a specified port. The port is defined during object construction, generally it should be random, and \verb!Server! will throw an \verb!IOException! if the port is already taken.
    
    \subsubsection{Session}
    
    The \verb!Session! and \verb!SessionIndex! classes form the basis around managing connections between servers -- a \verb!Session! object represents a single connection between two entities. \verb!Session! objects store a number of bits of information regarding a connection that provide context, such as the connection ID (for use in sending messages to the correct GUI window), the \verb!java.net.Socket! associated with the connection, the stream objects for sending/receiving data, statuses, and so on. A \verb!Session! object is created upon connection and filled with the relevant data (random connection ID, user details, etc.) and as long as the connection remains open the \verb!Session! object will remain in memory.
    
    However, Java objects are transient and cannot be easily transmitted across a network, particularly using XML. The \verb!Server! maintains an objects called \verb!SessionIndex!, which is a wrapper around a \verb!java.util.Hashtable! where the key is the connection ID (also referred to as a session ID) and the value is the corresponding \verb!Session! object. By doing this, we are able to maintain a persistent store of information regarding a connection that is easily accessible from all objects within the \verb!Server!, like packet handlers.
    
    The \verb!Session! is also used to store, upon connection, a \verb!Conversation! object for the newly opened conversation window. Upon receipt of a \verb!<message>!, the message handler will look up the session using the session ID received with the packet, and use methods within the \verb!Conversation! object to update the window with the received message.
    
    As local \verb!Session! objects are created by the server when a outbound connection attempt is made, they are used to handle the assembly and transmission of packets, such as connection packets, message, authentication, etc.
  
  \subsection{Protocol Model}
  
  \emph{For structural information regarding the protocol, see \textsection \ref{AppendixB}.}
  
  The protocol model is based around using XML (eXtensible Markup Language) to transmit data along with metadata (attributes) to represent instructions to be received by another system.
  
    \subsubsection{Routing and How It Works}
    
    We will find out in this section how packets are routed through the server upon receipt, along with how they are sent. Figure \ref{fig:conv_overview} provides an overview of how two entities will communicate with one another, and what packets are required.
    
    \begin{figure}
      \centering
      \includegraphics[scale=0.7]{./Figures/Ch6/6-4-3-1.pdf}
      \caption{The basic conversation elements between two entities.}
      \label{fig:conv_overview}
    \end{figure}
    
  \subsection{Parsing XML}
  
  Before we can start converting received messages in to Java objects and handling them, we must parse the XML in to something more usable.
  
    \subsubsection{SAX and Xerces}
    \label{subsubsec:sax}
    
    An implementation of the Simple API for XML (SAX) known as Apache Xerces is used as the SAX parser for incoming packets. Originally a Java-only library, it has become the standard for XML parsing within Java. It uses an event-driven model to parse XML documents, and rather than handling entire documents at once, it is done in a continuous stream of pieces, unlike DOM parsers (Document Object Model) which run using trees based on a complete document. The benefits of this for us are three-fold:
    
    \begin{enumerate}
      \item It is considerably faster than standard parsing, and works in linear time.
      \item Only small portions of a document need be in memory at any one time, and so the memory footprint of parsing a document is small.
      \item It can handle not being in possession of a full, valid XML document  when parsing, like the packets we will be sending between entities.
    \end{enumerate}
    
    It has one primary downside however: developer time. The concept of event-driven XML parsing is harder to manage than simply imagining a document as a tree. However, using Apache Xerces significantly improves upon this as the developer no longer has to handler the parsing itself, but merely what to do with the data once it is parsed. Xerces is more accurately a modular library for parsing, validating and manipulating XML documents. It introduces a framework known as \emph{Xerces Native Interface} that allows developers to build their own parser components, however we will only be using it for its SAX features.
    
    SAX parsers will follow through a document, and upon reaching specific points it will generate events and execute a callback function. For example, upon reaching the start of an element -- e.g. \verb!<message>! -- the parser could be configured to call a function \verb!startElement()! that determines what to do with this new element.
    
    One interesting thing to note is the use of the class \verb!StreamingCharFactory!. While capable of handling subsections of XML documents, SAX parsers are not generally (by default) able to handle streaming data as it arrives, and so the parsers will hang until reaching the end of the document (in this case, the connection closes). \cite{Shigeoka:2002ys} details the \verb!StreamingCharFactory! class that configures Xerces to use a streaming data reader rather than the default buffered one.
    
    \paragraph{The InputHandler} class deals with XML parsing by processing these SAX events. The primary way that documents will be handled is through \emph{depth} -- each section of the document is treated as a tree (as in DOM parsing). Upon encountering each element, the depth is increased by one, and decreased by one when the element ends. This is known as depth-first searching. We have three main methods that handle events:
    
    \begin{enumerate}
      \item \verb!startElement([...])! -- aptly named method that handles the start of elements.
      \item \verb!characters(char[] c,int start,int length)! -- handles the string values of elements.
      \item \verb!endElement(String uri,String localName,String name)! -- handles the closing tags for the current open element. 
    \end{enumerate}
    
    Methods 1 and 2 create \verb!Packet! objects that represent their respective XML elements. In the case of \verb!startElement()!, when at depth 0 the only element possible is a root \verb!<stream>! and so a \verb!Packet! with element name "stream" is created and pushed on to the queue, along with the attributes passed down from the SAX parser: \\
    
    \begin{lstlisting}
    public void startElement(String namespace,
                                 String localName,
                                 String name,
                                 Attributes attributes)
                     throws SAXException {
  switch (depth++){
    case 0: // Root element
      if (name.equals("stream")){
        Packet packet = new Packet(null,name,namespace,attributes);
        packet.setSession(session);
        packet_queue.push(packet);
        return;
      }
      throw new SAXException("Root element must be <stream>");
    case 1: // Message elements
      packet = new Packet(null,name,namespace,attributes);
      packet.setSession(session);
      break;
    default: // Any child elements
      Packet child = new Packet(packet,name,namespace,attributes);
      packet = child;
  }
}
    \end{lstlisting}
    
    As can be seen, at depth 1 a \verb!Packet! is created with all the provided attributes from the parser, including the name. At any other depth, however, we can assume we are dealing with a child element and so the previous \verb!Packet! is set as its parent (see \textsection\ref{subsubsec:packets} for details about how parent-child relationships are implemented programmatically) and all other attributes are stored as usual. At each depth, a global \verb!Packet! object is stored in the class to give access child packets access to parents.
    
    If we are dealing with a value of an element, for example the body of a message, the method \verb!characters()! is called by the parser. \verb!characters()! is a simple method that adds a child value to the last traversed packet in the tree. \\
    
    \begin{lstlisting}
public void characters(char[] c,int start,int length) throws SAXException {
  if (depth > 1) packet.getChildren().add(new String(c,start,length));
}
    \end{lstlisting}
    
    Finally, when an end element is reached we will once again check the current depth. If it is of depth 0, we must be at the end of the document and so a \verb!</stream>! packet is created and pushed on to the queue. At depth 1, the current packet is complete and can be pushed on to the queue, and otherwise we must still be assembling the parent, and so set the pointer to the parent packet. \\
    
    \begin{lstlisting}
public void endElement(String uri,
                            String localName,
                            String name)
       throws SAXException {
  switch(--depth){
    case 0: // End of the stream
      Packet c_packet = new Packet("/stream");
      c_packet.setSession(session);
      packet_queue.push(c_packet);
      break;
    case 1: // Put the completed packet on the packet queue
      packet_queue.push(packet);
      break;
    default:  // Parent still being constructed; traverse back up the tree
      packet = packet.getParent();
  }
}
    \end{lstlisting}
    
    \subsubsection{Packets}
    \label{subsubsec:packets}
    
    All packet related classes can be found in \emph{com.cyanoryx.uni.enigma.protocol.xml}.
    
    Packets in the context of the Enigma protocol are somewhat distinct from what are traditionally known as packets in networking. When we refer to packets we are referencing the XML sections representing messages sent between entities. In the previous section, \textsection\ref{subsubsec:sax}, we defined how these XML fragments are parsed and converted in to packets without really defining what a packet in this context is (henceforth any references to ``packet" will refer to the concept of sections of XML, unless stated otherwise or defined as a class, e.g. \verb!Packet!). Previously we described creating packets by instantiating \verb!Packet! objects with the name, attributes and child elements of the XML -- this class serves two purposes:
    
    \begin{enumerate}
      \item To provide the capability of storing packets internally and in memory without the need to re-parse each time the data is needed, and to provide helper methods to retrieve this data.
      \item To allow the simple, programmatic creation of XML packets through the instantiation of \verb!Packet! objects which can then be converted in to XML to be transmitted across the network.
    \end{enumerate}
    
    \begin{figure}
      \centering
      \includegraphics[scale=0.7]{./Figures/Ch6/6-4-4-2.pdf}
      \caption{The hierarchy of 3 packet objects.}
      \label{fig:packet_hierarchy}
    \end{figure}
    
    Figure \ref{fig:packet_hierarchy} displays how \verb!Packet! classes represent XML elements and store their children in what is effectively a tree. The corresponding XML would be:
    
    \begin{verbatim}
<message>
  <thread>12</thread>
  <body>Hi Bob, it's Alice</body>
</message>
    \end{verbatim}
    
    The \verb!Packet! class can be simply defined overall as a data structure consisting of a name, and attributes, with a \verb!java.util.LinkedList! of child \verb!Packet!s and a pointer to a parent \verb!Packet!, where applicable.
    
    Packets are parsed and handled on a first come, first served basis. Upon receipt, the \verb!Server! places packets in to a \verb!PacketQueue! where they will be picked out one by one by a running \verb!QueueThread! and handled by an appropriate class. \textsection\ref{subsec:combined} provides an overview of how \verb!Packets!, queues and thread fit together in the overall system.
    
    \verb!Packet! can be found in \emph{com.cyanoryx.uni.enima.net.protocol.xml.Packet}.
    
    \subsubsection{Packet Handling}
    \label{subsubsec:packet_queue}
    
      \paragraph{The PacketQueue} class is a thread-safe (synchronized) class that acts as a wrapper around a \verb!java.util.LinkedList! for storing all incoming packets. It has two methods, \verb!push! and \verb!pull!, the latter removes and returns the packet at the top of the queue, and the former adds a packet to the bottom of the queue. Both, alongside being implemented as synchronized, are thread-safe in that \verb!push! notifies all threads awaiting use of the queue when it is finished add the current packet, and \verb!pull! forces threads to wait if the queue is empty.
      
      \paragraph{The PacketListener} class, or more accurately the interface, provides a notify method implemented by the event handlers (\textsection\ref{subsubsec:handlers}) which allows the \verb!QueueThread! (below) to notify said handlers when a packet is pulled from the queue.
      
      \paragraph{The QueueThread} class is perhaps the most important as it acts as the link between receiving a message and handling it. A \verb!QueueThread! is started when a server is first created, and is passed all the desired event handlers and their XML tag identifiers, which are stored in a \verb!java.util.HashMap!. For example, the \verb!OpenStreamHandler! is required to handle new connection requests, signified by \verb!<stream>! and as such its identifier is "stream." The \verb!QueueThread! is constantly running alongside the server, looking for new packets pushed on to the queue. When it notices a new packet, it pulls it out of the queue and checks the element name of the packet and looks for it in its handler \verb!HashMap!. If a handler is found, it is notified and passed the \verb!Packet! object.
      
      The idea for this came from \cite{Shigeoka:2002ys}.
      
      \begin{lstlisting}
  public void run(){
    for (Packet packet=packetQueue.pull();packet!=null;packet=packetQueue.pull()) {
      try {
        // Element name to look for.
        // Matches name of listener
        String match = packet.getElement();

        synchronized(packetListeners){
          Iterator<PacketListener> iter = packetListeners.keySet().iterator();
          // Loop through the current set of listeners
          while (iter.hasNext()){
            PacketListener listener = iter.next();
            // Get the name of the tag to match
            String listenerString = packetListeners.get(listener);
            // If the packet's element matches the element for this listener...
            if (listenerString.equals(match)){
              listener.notify(packet); // ..send packet to handlers
            } 
          } 
        } 
      } catch (Exception e){
        e.printStackTrace();
      }
    } 
  } 
      \end{lstlisting}
      
      \verb!QueueThread! allows many packets to be sent to a server at once, with each being handled in order and none being lost.
    
      \paragraph{The ProcessThread} class is created whenever a new connection is made to the \verb!Server!. As \verb!InputHandler! objects parse an entire XML document, they are tied-up with each connection until the document is complete (i.e. a \verb!</stream>! is sent). Because of this, if we were to use one \verb!InputHandler! per server, only one connection would be possible at any one time. As a solution, \verb!ProcessThread! is created for each connection, which instantiates a new \verb!InputHandler! object and begins the XML processing on the incoming stream.
    
    \subsubsection{Handlers}
    \label{subsubsec:handlers}
    
    Handler classes are where most of the work is done once a packet has been parsed. Four are required for the basic operation of the Enigma protocol:
    
    \begin{enumerate}
      \item \textbf{OpenStreamHandler} -- handles an open stream request and initiates some default settings such as whether or not the conversation will be authenticated -- \verb!<stream>!.
      \item \textbf{CloseStreamHandler} -- gracefully handles closed connections and cleans up \verb!Session! objects -- \verb!</stream>!.
      \item \textbf{AuthHandler} -- handles all authentication and encryption setup, including received public keys, session keys and so on -- \verb!<auth></auth>!.
      \item \textbf{MessageHandler} -- handles all incoming messages, determines who they are from and displays them in the appropriate window -- \verb!<message></message>!.
    \end{enumerate}
    
    Handlers implement the \verb!PacketListener! interface, meaning they are notified whenever the matching packet type is parsed out of the queue. See \textsection\ref{subsubsec:packet_queue} for further details about how \verb!QueueThread! determines which handler to use. Handlers, on creation, also receive a reference to the master \verb!SessionIndex! within \verb!Server!.
    
    The handler implementations can be found at \emph{com.cyanoryx.uni.enigma.server}.
    
    \subsection{Fitting the Protocol Components Together}
    \label{subsec:combined}
    
    So, how do these classes fit together in the final system? We've so far given distinct diagrams, and provided abstract descriptions of classes and how they relate, but this is hard to conceive as a whole system.
    
    \begin{center}
      \includegraphics[scale=0.5]{./Figures/Ch6/6-4-5.pdf}
    \end{center}
    
    This figure shows the process of \verb!Server! creation, through to client connection and and message receipt.

\section{Algorithm Implementation}
  \emph{Packages covered in this section: com.cyanoryx.uni.crypto.*}
  
  \subsection{Interfaces and Abstraction}
  
  One of the major points of the Enigma application is the ability to easily introduce new symmetric or asymmetric algorithms in to the application. To do this, we use a well known concept of object-oriented programming: abstraction. Abstraction, in its basic form, is the process of representing some data or logic but hiding the actual implementation. A good example of this is our \verb!LogHandler! earlier, which provides an abstraction of outputting log information -- the developer/program is provided with an interface with, for example, a \verb!log()! method that displays a supplied string in a log window. The implementation of the log function is irrelevant to the developer, all they need to know is that they can use it to reliably log information. Because of this, the logic behind logging data can be changed as necessary assuming it still has the same output -- this is known as a "black box."
  
  In Java, abstraction is implemented using \emph{interfaces}. An interface defines methods along with their input and output data types, and any class that \emph{implements} this interface must create the logic for these methods. How are interfaces applicable to Enigma? Producing a program that encrypts data using multiple libraries can be difficult as each may have subtle differences in how they expect to receive and output data. For example, a block cipher $A$ may implement a \verb!encrypt()! method that takes portions of data as byte arrays, whereas a cipher $B$ may take the entire plaintext as \verb!String!. To solve this, an interface will be introduced with an \verb!encrypt! method that requires all implementing classes use it -- this results in a standard implementation of all algorithms, meaning a developer knows without a doubt what to input and what to expect back.
  
  This is particularly applicable to Enigma as it means code repetition when dynamically switching between algorithms is reduced significantly. Alongside this, if a new algorithm is desired for use, it simply needs to implement the Enigma encryption interface.
  
  \subsection{Key Agreement and Certficates}
  
  The initiator of a conversation sends their certificate (including their public-key signed by a Certificate Authority) immediately after a successful connection is made. If the certificate is valid and verifiable, the receiver will generate a session key and encrypt it using the initiator's public key before transmitted it. Both users now have a shared session key, and can begin encrypting messages before sending.
  
  This is the standard flow for authentication and key agreement. This takes place primarily in \verb!Connect#connect()! and \verb!AuthHandler#notify()!, which can be found in \emph{com.cyanoryx.uni.enigma.gui} and \emph{com.cyanoryx.uni.enigma.net.server}, respectively.
  
  \paragraph{Key Generation} is performed through an option in the toolbar -- Options > Generate Keys... -- and requires no user input other than having previously set the Certificate Authority's key location in the preferences.
    
  This utility automatically generates a public/private key pair for the user and then signs it with the Certificate Authority's private key to produce their own certificate which will be used by other entities to confirm their identities. In the Real World, user's would not have access to the private keys of the CA, however as this application is for research and will not utilise an actual CA we will assume the ownership of CA private keys.
  
  \subsection{Ciphers}
  
  Now that we have a shared session key, messages can now be encrypted by either entity and transmitted over the network securely. 
  
  \paragraph{Sending a message} is handled by \verb!Session#sendMessage()! which will, before sending a message, check to see if the current connection requires encrypted messages and if so it encrypts them using the chosen algorithm. For example, AES: \\
  
  \begin{lstlisting}
AES aes = new AES();
				
Key k = new Key(KeySize.K256);
k.setKey(key);
aes.setKey(k);
			
aes.setPlainText(body.getBytes());
msg=Base64.encodeBytes(aes.encrypt());
\end{lstlisting}

  As can be seen, messages are encoded using Base64 before transmission. Symmetric ciphers encrypt in blocks of bytes, and so trying to send the output from a cipher directly will result in sending binary data which can cause issues within XML documents. Base64 is an encoding scheme that takes binary data and represents it as ASCII text, meaning it can be printed and displayed using almost all character sets. The inner workings of Base64 are out of the scope of this project, however \textsection\ref{AppendixD} lists the open source library used for encoding and decoding.
  
  It should be noted that the actual XML of a message is not encrypted, but only the body.
  
  \paragraph{Receiving a message} is simple, first we must check to see if we are expecting and allowing encrypted messages, and if so create the appropriate cipher object. For example, using AES: \\
  
  \begin{lstlisting}
AES aes = new AES();
Key k = new Key(KeySize.K256);
// Get the session key from the Session object
k.setKey(s.getCipherKey());
aes.setKey(k);
// Convert the Base64 encoded message to binary
aes.setCipherText(Base64.decode(message));
// And decrypt it, producing a String object from the 
// byte array
message = new String(aes.decrypt());
\end{lstlisting}

  The message will then, as usual, be sent to the appropriate conversation window.

\section{Summary}

We have now listed all the general components that make up the Enigma application: interface, server and algorithms. We will now go on to think about the algorithms and their implementations in further detail -- known as cryptanalysis.

\section{Usage}

  A user's instruction manual is included as \emph{\textsection \ref{AppendixD}}. This manual also briefly covers compilation, required dependencies, and other build related information.

\section{Program Listing}

  Due to the size of the project a detailed, printed code listing is impossible, and thus it is recommended that the code be viewed using a text editor or other text environment on a device. If you do not have a digital copy of this project, please see \emph{\textsection \ref{sec:project_repo}}. 

% Chapter 7

\chapter{Cryptanalysis} 
\label{Chapter7}
\lhead{Chapter 7. \emph{Cryptanalyis}} 

\section{Public-key Cryptography}

  \subsection{RSA}
    
    A distinction must be made between the security of the RSA \emph{algorithm} and the RSA \emph{system}. This is known as semantic security \cite{Goldwasse:1990aa}, the use of other non-algorithmic techniques to resist attacks and make it, for example, difficult to recover information from the public key. As we discussed in Chapter 3, standards exist such as RSA-OAEP that introduce elements of randomness in to the RSA algorithm that limit certain basic attacks -- this is the RSA algorithm implemented in to a cryptosystem. The RSA algorithm itself is distinct from this at is the core mathematics, rather than an implementation of a secure system.
    
    In this section, we will discuss both attacks on the RSA \emph{algorithm} and attacks against RSA \emph{systems}, such as the one we have implemented.
    
    \subsubsection{Brute Force}
    
      The first attack we should mention is known as the \emph{brute-force} attack: factorising the integer modulus $N$ to determine the prime numbers used to generate the keys. This has been discussed throughout the paper, and due to the simplicity (to an extent) of the attack, we will not extensively cover it. It is interesting to note that there are no (publicly available) efficient algorithms with an acceptable running time for factoring large prime multiples. The current fastest algorithm is the \emph{General Number Field Sieve}, which runs on $\exp\left( \left(\sqrt[3]{\frac{64}{9}} + o(1)\right)(\log n)^{\frac{1}{3}}(\log \log n)^{\frac{2}{3}}\right) =L_n\left[\frac{1}{3},\sqrt[3]{\frac{64}{9}}\right]$ time, with an $n$-bit integer. See \cite{Briggs:1998aa} for an interesting introduction to GNFS.
      
      If an algorithm for factoring large integers in reasonable time periods is ever discovered, RSA will be rendered entirely insecure. However, until then we will only consider attacks that can decrypt RSA ciphertext \emph{without} factoring the modulus $N$.
    
    \subsubsection{Elementary Attacks}
    
      \paragraph{Small Exponent}
  
        \subparagraph{Private Exponent}
        \subparagraph{Public Exponent}
        
          It is common belief that using a small encryption exponent $e$ does not adversely affect the capabilities of the RSA algorithm. For example, it is suggested $e=3$.
          
          This is best shown as an example. If Alice wishes to send a message $m$ to three associates with the public moduli $n_{\{1,2,3\}}$ and exponent $e=3$, she would send $c_i = m^3$ mod $n_i$. Through a eavesdropping, an attacker Mallory could use Gauss's Algorithm all three values of $c$ to find:
          
          \[
          	\left\{
          	\begin{array}{ll}
                    x \equiv c_1 \  mod \  n_1 \\
                    x \equiv c_2 \  mod \  n_2\\
                    x \equiv c_3 \  mod \  n_3
                  \end{array}
                  \right.
          \]
          
          Using the Chinese Remainder Theorem, we can determine that $x = m^3$, and thus $m = \sqrt[3]{x}$. Also, if a message $m < n^{\frac{1}{e}}$ we can calculate the $e^{th}$ root of $(c=m^e)$ to get the plaintext message.
          
          It should be noted that this attack cannot be considered a "break" of the algorithm. Both these attacks can be mitigated through the use of salting -- adding randomly generated bits to messages before encryption. Another simple, but not recommended, solution is to make $e \geq 2^{16}+1$.
                    
          Due to the use of hashing and salting, this attack does not affect RSA cryptosystems such as RSA OAEP, the system implemented for Enigma.
        
      \paragraph{Key Exposure}
      \paragraph{Coppersmith's Short Pad Attack}
        \subparagraph{Hastad's Broadcast Attack}
      \paragraph{Related Message Attack}
      \paragraph{Forward Search Attack}
      \paragraph{Common Modulus Attack} 
      
        A previously common solution to managing the keys of multiple entities within one network was creating a central authority that would select a modulus $N$, and then share exponent pairs $e_i$ and $d_i$ with the entities. While this attack relies on the ability to factor prime multiples, given $(e_i,d_i)$, an attacker could determine the decryption exponents for all other entities within the network. 
      
      \paragraph{Cycling Attack}
      
        	
      
      \paragraph{Message Concealing}
    
    \subsubsection{System and Implementation Attacks}
      \paragraph{Timing}
      \paragraph{Random Faults}
      \paragraph{Bleichenbacher's Attack}
  
  \subsection{Certificates and Authentication}
  
    \subsubsection{Implementation Attacks}
    
      \paragraph{SSL BEAST}
    
      \paragraph{Authority Security and Impersonation}
      
        The underlying security of the certification system is based upon a trusted third party (TTP) that verifies the chain of trust -- the certificate authority (CA) -- and so ``a chain is only as strong as its weakest link" appears to be quite apt. If, through social or technical means, an attacker can gain access to the private keys or construction information for the private keys of a CA, they will be able to issue certificates as they wish until the breach is noticed and the CA can be removed from the chain.
    
        This is not a theoretical attack by any means. In July 2011, Mozilla -- the developers of many open source products such as Firefox and Thunderbird -- was informed that a fraudulent SSL certificate belonging to Google, Inc. had been issued by CA \emph{DigiNotar}. As it happens, DigiNotar's network had been breached allegedly without their knowledge, giving the attacker access to their private keys and thus allowing certificates to be issued arbitrarily. In this case, it was shown to be used by unknown entities in Iran to conduct a man-in-the-middle against Google services \cite{Google:2011ah}.

\section{Symmetric Cryptography}

  This sections covers attacks affecting the AES symmetric cipher.

  \subsection{Brute-force}
  \subsection{XSL Attack}
  \subsection{Biryukov and Khovratovich}
  \subsection{Related Key Attack}
  \subsection{Known-key Distinguishing Attack}
  \subsection{Bogdanov, Khovratovich, and Rechberger}
  \subsection{Side-channel attacks}

\section{Hash Functions}

  \subsection{Birthday Attacks}
  \subsection{Collision and Compression}
  \subsection{Chaining Attack}

\section{Emerging Threats}

  \subsection{Quantum Cryptography and Cryptanalysis}
  
    Quantum computing is an emerging technology.
    % Bits about quantum computing
    
    % Shor's algorithm
    
    As well as affecting algorithms based around the intractability of integer factorisation, quantum cryptanalysis also affects algorithms that utilise the discrete logarithm problem such as ElGamal, Diffie-Hellman and DSA.
    
    Quantum cryptography, in its current form, does not affect modern symmetric cryptographic algorithms -- neither ciphers or hash functions. Grover's algorithm, an algorithm that improves the efficiency of searching an unsorted database, improves the speed at which a symmetric cipher key can be cracked, however this is preventable using standard counter-actions such as increasing the key size. 
    
    An interesting current field of study is post-quantum cryptography: algorithms designed such that a cryptanalyst with a powerful quantum computer (should they become prevalent, or even plausibly workable in the future) cannot easily break. See \cite{Bernstein:2009aa}.
  
  \subsection{Ron was wrong, Whit is right}
  
    In early February 2012, a paper entitled ``Ron was wrong, Whit was right" -- a slight jab at RSA co-creator Ronald Rivest and nod towards cryptographer Whitfield Diffie -- swept through the headlines of the cryptographic communities. Written by researchers at the \emph{School of Computer and Communication Studies, \'{E}cole Polytechnique F\'{e}d\'{e}rale De Lausanne}. The paper detailed a "sanity check" of a subsection of RSA public keys that can be found online, and analysed them to test the randomness of the inputs used to calculate the keys.
    
    As we are now well aware, RSA is based entirely on the inability to efficiently factor prime numbers. However, there are other requirements for the good security of the algorithm. Namely, it is crucial that when the keys are generated, previous random numbers are not reused. \cite{Loebenberger:2011aa} states that: 
    
    Given a study of 11.7 million public keys, 
    
    Conversely, the researchers were unable to find any of the common exponents, used in RSA public keys, being used for the other two major public-key algorithms: DSA and ElGamal. ECDSA was also investigated, however only one certificate was found to be using ECDSA.
    
  \subsubsection{Sony PlayStation 3}
  
    An unrelated, yet high-profile, realisation of the risks of poor entropy in random number generators is the cracking of Sony's PlayStation 3 console. Sony used public-key cryptography to sign its bootloaders and games, which prevents unauthorised software being executed on the device. The overall system used to protect the device consists of many different techniques and algorithms, however it was the implementation of \emph{Elliptic Curve Digital Signature Algorithm (ECDSA)} that was flawed. As we will show below, a poorly executed random number generator design led to the extraction of the private key stored securely within the device, previously implausible to retrieve.
    
    ECDSA uses 11 parameters in its cryptographic algorithms: 9 public, and 2 private:
    
    \begin{center}
    \textbf{Public}
    
    $p, a, b, G, N =$  curve parameters
    
    $Q =$ public key
    
    $e =$ data hash
    
    $R, S =$ signature
    
    \textbf{Private}
    
    $m =$ random number
    
    $k =$ private key
    
    \end{center}
    
    The signature, $R, S$ is computed:
    
    \begin{center}
    
    $R = (mG)_x$
    
    $S = \frac{e+kR}{m}$
    
    \end{center}
    
    Because of this, it is absolutely vital that a high-entropy random number generator be used to produce $m$, otherwise given two signatures with the same $m$, we are able to determine $m$ \emph{and} private key $k$:
    
    \begin{center}
        \begin{tabular}{ c c }
          $R = (mG)_x$ & $R = (mG)_x$ \\
          $S_1 = \frac{e_1+kR}{m}$ & $S_2 = \frac{e_2+kR}{m}$
        \end{tabular}
    \end{center}
    
    Rearranging:
    \begin{center}
    	$S_1 - S_2 = \frac{e_1-e_2}{m}$
    
    	$m = \frac{e_1-e_2}{S_1-S_2}$
    
    	$\therefore k = \frac{e_1S_2-e_2S_1}{R(S_1-S_2)} = \frac{mS_n-e_n}{R}  $
    \end{center}
    
    Given $k$, we have now have the capability to sign arbitrary data, meaning: the chain of trust is broken, signed executables are no longer useful, encrypted storage can be accessed, and many other features of the security system rendered ineffective \cite{Bushing:2010qs}.
    
    While the human-impact was virtually zero in terms of loss of life, injury, etc. this is an excellent example of how something seemingly easy to implement like an RNG can affect the overall security of an entire system and the impact it can have.

\section{A comment on theory}

It is interesting to note, based upon history, that in all likelihood these theoretical attacks are currently being implemented. Though 

However, the topic of real-world security flaws and those that take advantage of them is extensive and could easily be covered by its own paper, if not several. It is left up to the reader's imagination to consider the many possible vulnerabilities currently being utilised unbeknownst to the vast majority of the community, and also to forget this idea lest they never use a networked device again.

% Chapter 8

\chapter{Outcomes and Further Research} 
\label{Chapter8}
\lhead{Chapter 8. \emph{Assessment and Further Research}} 

\section{Fulfilment of Specification}

All points in the specification (see \textsection\ref{AppendixA}) were met fully, and the requirements have been fulfilled entirely.

\section{Comparison to Standard Libraries}

  \subsection{Asymmetric Cryptography}
  
    \begin{center}
      \begin{tabular}{ | l | l |}
        \hline
        Library & Average Time (20 iterations; nanoseconds) \\ \hline \hline
        Enigma &  \\ \hline
        Cryptix &  \\ \hline
        BouncyCastle & \\ \hline
        JDK & \\ \hline
        \hline
      \end{tabular}
    \end{center}
  
  \subsection{Symmetric Cryptography}
  
    \begin{center}
      \begin{tabular}{ | l | l |}
        \hline
        Library & Average Time (20 iterations; nanoseconds) \\ \hline \hline
        Enigma &  \\ \hline
        Cryptix &  \\ \hline
        BouncyCastle & \\ \hline
        JDK & \\ \hline
        \hline
      \end{tabular}
    \end{center}

\section{Summary}

Our aim in this project was to produce, from scratch and using only technical documentation, implementations of cryptographic algorithms that could be used in a real world application. This objective was completed successfully, and a large amount of new knowledge has been gained. However, the final recommendation? Do not do this. Cryptographic implementations have long been standardised and produced in mature libraries that are easy to use.

The main point is this: if you are doing something so unusual with cryptography such that you need to implement algorithms yourself, then you shouldn't be doing that. If you aren't doing something unusual, then you can use the libraries already available.

This brings us back to the quote from Bruce Schneier given at the beginning of this report: 

\textit{``Anyone can design a security system that he cannot break. So when someone announces, ``Here’s my security system, and I can`t break it,” your first reaction should be, “Who are you?” If he`s someone who has broken dozens of similar systems, his system is worth looking at. If he`s never broken anything, the chance is zero that it will be any good.''}

Non-standard and custom implementations are only truly safe when they have been created by developers experienced in the production of cryptosystems and tested thoroughly.

The author hopes that this report has been of interest to you, or at least beneficial to your knowledge of information security, cryptosystems and how difficult it really is to get it right.

% Chapter 7

\chapter{Further Research} % Write in your own chapter title
\label{Chapter7}
\lhead{Chapter 7. \emph{Further Research}} % Write in your own chapter title to set the page header

????

\input{./Chapters/Chapter10}

%% =====================
% APPENDICES

\addtocontents{toc}{\vspace{2em}} 

\appendix 

% Appendix A

\chapter{Enigma Application Software Requirements Specification}
\label{AppendixA}
\lhead{Appendix A. \emph{Enigma Application Specification}}

\section{Introduction}

This document lists basic requirements for the Enigma application. All these conditions must be met for the proper working of the application and for it to work as expected. This is not a high-level design specification, but an overview.

\subsection{Scope}

This document applies to the overall Enigma application that will function as an instant-messaging server. The individual cryptographic algorithms are not specified here.

\section{Overall Description}

The purpose of the Enigma application is to provide a test bed for public-key and asymmetric encryption algorithms so that they can be tested in a real-world use environment and evaluate performance using external tools.

\subsection{Product Functions}

\subsubsection{Primary Functions}

\begin{itemize}
  \item To connect to and handshake with a remote Enigma server.
  \item To receive and accept connection requests from remote Enigma servers.
  \item To send ASCII text messages between two entities connected through Enigma servers.
  \item To be able to ``toggle" encryption of messages on/off, allowing for secure and insecure sessions.
  \item To use public-key cryptography to securely agree on a shared session key for use in symmetric ciphers.
  \item To securely encrypt and decrypt messages given a designated cipher function and a key.
  \item To verify the identity of a remote entity using a public-key certificate.
\end{itemize}

\subsection{Interfaces and Accessibility}

The application will have a graphical user interface (GUI) that will use the native components of the host operating system.

\subsection{User Characteristics}

The user will likely be:

\begin{itemize}
  \item A developer or someone who is knowledgeable about computer systems.
  \item A person who is connected to a developer and is aiding in testing/use for analysis, and thus has instructions for use.	
\end{itemize}

\subsection{Constraints}

The software should \textbf{not} be considered for use in an environment where sensitive information is required to be shared which, if released, would pose a threat to life, cause injury or damage to persons or property, or any other result which could be considered damaging or illegal.

No guarantees of running-time can be made, and neither the encryption, decryption or transmission of data should be considered real-time.

The software must be used purely for research purposes.

\subsection{Assumptions and dependencies}

\begin{itemize}
  \item The host computer(s) are capable of running at least Java 6.
  \item The source must be compiled with the appropriate library dependencies.
  \item Two entities wishing to communicate must be connected via a network, be it a local connection or wider (such as the internet).
\end{itemize}

% Appendix B

\chapter{Enigma Protocol Specification}
\label{AppendixB}
\lhead{Appendix B. \emph{Enigma Protocol Specification}}

\section{Introduction}

\section{Terminology}

\section{Definitions}

\section{Format}

\section{Commands}

\subsection{Connection}

A connection is represented by a streaming XML document, with the root element being \verb!<stream>!.

\begin{itemize}
	\item \textbf{Opening a connection:}
		\begin{verbatim}
			<stream  to="SERVER_NAME"
				         from="USER_NAME"
				         id="SESSION_ID"
				         return-port="LOCALHOST_INBOUND_PORT"
				         xmlns="enigma:client">
		\end{verbatim}
	\item \textbf{Closing a connection:}
		\begin{verbatim}
			</stream>
		\end{verbatim}
\end{itemize}

\subsection{Authentication}

\begin{itemize}
	\item \textbf{Toggling encryption:}
		\begin{verbatim}
			<auth stage="streaming"
			      id="SESSION_ID"
			      type="toggle">
			      [off|on]
			</auth>
		\end{verbatim} \\
		Note: this requires the agreement of both users. The connection should be closed if one user disagrees to change the current encryption status.
\end{itemize}

\begin{itemize}
	\item \textbf{Asserting the key agreement method:}
		\begin{verbatim}
			<auth stage="agreement"
			      id="SESSION_ID"
			      type="method">
			      [method type identifier]
			</auth>
		\end{verbatim} \\
\end{itemize}

\begin{itemize}
	\item \textbf{Publishing a certificate:}
		\begin{verbatim}
			<auth stage="agreement"
			      id="SESSION_ID"
			      type="cert">
			      [Base64 encoded Enigma Certificate]
			</auth>
		\end{verbatim} \\
\end{itemize}

\begin{itemize}
	\item \textbf{Publish an encrypted symmetric cipher key:}
		\begin{verbatim}
			<auth stage="agreement"
			      id="SESSION_ID"
			      type="key"
			      [method="CIPHER_ALGORITHM"]>
			      [Base64 encoded encrypted key]
			</auth>
		\end{verbatim} \\
\end{itemize}

\subsection{Messaging}

\begin{itemize}
	\item \textbf{Sending a message:}
		\begin{verbatim}
			<message  [to="REMOTE_USER_NAME"]
				          from="USER_NAME"
				          id="SESSION_ID"
				          [type=""]>
			    <body>MESSAGE_CONTENT</body>
			</message>
		\end{verbatim}
\end{itemize}

\subsection{Errors}

Errors do not have a specific element themselves, but are included as a subelement of any other tag, setting \verb!att:type! to \verb!error!.

\begin{itemize}
	\item \textbf{Sending an error:}
		\begin{verbatim}
			<element  [other attributes]
				          type="error">
			    <error type="ERROR_NUMBER">
			        ERROR_MESSAGE
			    </error>
			    [element contents]
			</element>
		\end{verbatim}
\end{itemize} 

\addtocontents{toc}{\vspace{2em}} 
\backmatter

%% =====================
% BIBLIOGRAPHY

\label{Bibliography}
\lhead{\emph{Bibliography}}  
\bibliographystyle{unsrtnat}  
\bibliography{Project}  % The references (bibliography) information are stored in the file named "Bibliography.bib"

\end{document}  % The End
%% ----------------------------------------------------------------